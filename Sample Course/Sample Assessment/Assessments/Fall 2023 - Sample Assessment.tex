\documentclass[12pt]{amsart}

\usepackage{fullpage, graphicx,multicol,fancyhdr,ifthen}
\usepackage[shortlabels]{enumitem}
\usepackage{tikz}
\usetikzlibrary{calc,patterns,angles,quotes,math,snakes}

\setlength{\parskip}{6pt}
\setlength{\parindent}{0pt}

\setlength{\textheight}{9in}
\setlength{\topmargin}{-0.75in}
\setlength{\textwidth}{6.5in}
\setlength{\rightmargin}{1in}
\setlength{\oddsidemargin}{-.2in}
\setlength{\parskip}{6pt}
\setlength{\parindent}{0pt}
\setlength{\headsep}{1cm}


%\pagestyle{empty}
\pagestyle{fancy}
\fancyhf{} 
\pagenumbering{gobble}

\begin{document}

\rhead{10001}\lhead{Sample Course}\chead{Sample Assessment - Fall 2023}\graphicspath{{C:/Users/iainc/anaconda3/Randomizer/Sample Course/Sample Assessment/}}\pagenumbering{arabic}\setcounter{page}{1}


\thispagestyle{fancy}

 \noindent Name: Newton, Isaac \vspace{.3cm} \\\noindent Student ID: 8675309 \vspace{.3cm} \\\noindent Instructor: J. Brennan \vspace{.3cm} \\\noindent Signature: $\rule{6cm}{0.15mm}$ \vspace{.3cm} \\ 



\vspace{.4cm}

\noindent {\bf Note that the 'assessmentpreface.tex' file in the exams archive folder is read and placed here. This is also where student information is included, either to be replaced with information from the master.csv file or as blanks.}

\vspace{.4cm}

\hrule

\subsection*{Instructions:} \begin{enumerate}[1.]
\item Any cover page materials, per your departmental standards.
\end{enumerate}


\newpage
\begin{enumerate}
\item {\bf (2 points)} 
 Which of these is correct?

\begin{minipage}[t]{1.0\linewidth}\begin{multicols}{4}\begin{itemize}\item[(a)]  Wrong. \item[(e)]  Wrong. \item[(i)]  Wrong. \item[(b)]  Wrong. \item[(f)]  Wrong. \item[(j)]  Wrong. \item[(c)]  Wrong. \item[(g)]  Wrong. \item[] \item[(d)]  Correct. \item[(h)]  Wrong. \item[] \end{itemize}\end{multicols}\end{minipage} \vfill 
\item {\bf (2 points)} 
 Which of these is correct?

\begin{minipage}[t]{1.0\linewidth}\begin{multicols}{3}\begin{itemize}\item[(a)]  Wrong. \item[(b)]  Correct. \item[(c)]  Wrong. \end{itemize}\end{multicols}\end{minipage} \vfill 
\item {\bf (2 points)} 
 Which of these isn't mentally problematic?

\begin{minipage}[t]{1.0\linewidth}\begin{itemize}\item[(a)]  None of the below.  \item[(b)]  $a \neq a$ \item[(c)]  I've built a set that contains itself. \item[(d)]   All of the above. \end{itemize}\end{minipage} \vfill 
\item {\bf (2 points)} 
 The least common denominator for $\displaystyle \frac{x}{x+1}$ and $\displaystyle \frac{1}{x(x-1)}$ is \vspace{.2cm}

\begin{minipage}[t]{1.0\linewidth}\begin{multicols}{2}\begin{itemize}\item[(a)]  $x(x+1)$ \item[(c)]  $(x+1)(x-1)$ \item[(b)]  $x(x+1)(x-1)$ \item[(d)]  $x+1$ \end{itemize}\end{multicols}\end{minipage} \vfill \newpage\def \a{7}\def \atwoone{1}\def \atwotwo{-6}\def \atwothree{2}\def \btwothree{7}\def \sumtwothree{9}\def \diftwothree{-5}\def \bigtwothree{200}\def \powtwothree{49}\def \logtwothree{0.3562071871080222}\def \factortwothree{95}\def \atwofour{1.68}\def \btwofour{1.156}\def \tooshorttwofour{10.1}\def \moneytwofour{10.10}\def \longertwofour{10.10000}\def \atwofive{0.12}\def \btwofive{0.12346}\def \athreeone{6}\def \bthreeone{8}\def \setthreetwo{[2, 5, 6]}\def \athreetwo{2}\def \bthreetwo{5}\def \cthreetwo{6}\def \controlthreethree{-8}\def \athreethree{2}\def \topthreethree{1}\def \athreefour{4}\def \bthreefour{3}\def \listthreefour{[1, 2, 3, 5]}\def \afourone{8}\def \bfourone{-6}\def \fracfourone{\frac{-4}{3}}\def \rootfourtwo{20}\def \simplifiedfourtwo{2 \sqrt{5}}\def \sqrtlistfourtwo{[2, 5]}\def \outfourtwo{2}\def \infourtwo{5}\def \wowfourtwo{1}\def \afourthree{-5}\def \nicethreefour{3x^{2}-x^{}-5}\def \nastythreefour{xyz^{3}-5}\def \cfourthree{-4}\def \dfourthree{9}\def \infourthree{-4x^{}}\def \outfourthree{+9y^{}}\def \afourfour{1350986}\def \nicefourfour{1,350,986}\def \goodfourfour{1,000,000.12345}\def \badfourfour{1,000,000.1}
\item {\bf (12 points)} 
 $a = \a$ 
\vfill 
\def \a{7}\def \atwoone{2}\def \atwotwo{5}\def \atwothree{1}\def \btwothree{6}\def \sumtwothree{7}\def \diftwothree{-5}\def \bigtwothree{100}\def \powtwothree{6}\def \logtwothree{0.0}\def \factortwothree{85}\def \atwofour{1.63}\def \btwofour{1.843}\def \tooshorttwofour{10.1}\def \moneytwofour{10.10}\def \longertwofour{10.10000}\def \atwofive{0.12}\def \btwofive{0.12346}\def \athreeone{4}\def \bthreeone{2}\def \setthreetwo{[12, 6, 9]}\def \athreetwo{12}\def \bthreetwo{6}\def \cthreetwo{9}\def \controlthreethree{-5}\def \athreethree{3}\def \topthreethree{0}\def \athreefour{4}\def \bthreefour{3}\def \listthreefour{[1, 2, 3, 5]}\def \afourone{4}\def \bfourone{-8}\def \fracfourone{\frac{-1}{2}}\def \rootfourtwo{12}\def \simplifiedfourtwo{2 \sqrt{3}}\def \sqrtlistfourtwo{[2, 3]}\def \outfourtwo{2}\def \infourtwo{3}\def \wowfourtwo{1}\def \afourthree{0}\def \nicethreefour{3x^{2}-x^{}}\def \nastythreefour{xyz^{3}}\def \cfourthree{4}\def \dfourthree{-10}\def \infourthree{4x^{}}\def \outfourthree{-10y^{}}\def \afourfour{1283078}\def \nicefourfour{1,283,078}\def \goodfourfour{1,000,000.12345}\def \badfourfour{1,000,000.1}
\item {\bf (12 points)} 
  \goodfourfour $\neq$ \badfourfour 
\vfill 
\def \a{7}\def \atwoone{3}\def \atwotwo{-6}\def \atwothree{1}\def \btwothree{7}\def \sumtwothree{8}\def \diftwothree{-6}\def \bigtwothree{100}\def \powtwothree{7}\def \logtwothree{0.0}\def \factortwothree{91}\def \atwofour{1.94}\def \btwofour{1.646}\def \tooshorttwofour{10.1}\def \moneytwofour{10.10}\def \longertwofour{10.10000}\def \atwofive{0.12}\def \btwofive{0.12346}\def \athreeone{6}\def \bthreeone{5}\def \setthreetwo{[3, 7, 7]}\def \athreetwo{3}\def \bthreetwo{7}\def \cthreetwo{7}\def \controlthreethree{-5}\def \athreethree{3}\def \topthreethree{1}\def \athreefour{5}\def \bthreefour{3}\def \listthreefour{[1, 2, 3, 4]}\def \afourone{12}\def \bfourone{4}\def \fracfourone{3}\def \rootfourtwo{8}\def \simplifiedfourtwo{2 \sqrt{2}}\def \sqrtlistfourtwo{[2, 2]}\def \outfourtwo{2}\def \infourtwo{2}\def \wowfourtwo{1}\def \afourthree{5}\def \nicethreefour{3x^{2}-x^{}+5}\def \nastythreefour{xyz^{3}+5}\def \cfourthree{4}\def \dfourthree{-10}\def \infourthree{4x^{}}\def \outfourthree{-10y^{}}\def \afourfour{1815827}\def \nicefourfour{1,815,827}\def \goodfourfour{1,000,000.12345}\def \badfourfour{1,000,000.1}
\item {\bf (12 points)} 
  $\sqrt{\rootfourtwo} = \simplifiedfourtwo$ 
\vfill 
\newpage\def \x{75}\def \y{150}\def \L{300}\def \area{11250}
\item {\bf (5 points)} 
 You have $\L$ feet of fencing to enclose a rectangular plot that borders on a river. If you do not fence along the side of the river, find the \textbf{dimensions} of the plot that will maximize the area. \\

\begin{tikzpicture}[scale=0.2]
    \draw[snake=coil,segment aspect=0,gray] (0,0)   -- (20,0);
    \draw[black] (4,0) -- (4,-4) -- (16,-4) -- (16,0);
    \draw[gray] (10,1.5) node {River};
    \draw[black] (16,-4) node[anchor=west] {Fence};
\end{tikzpicture}
  
\vfill \vfill \vfill
\def \a{5}\def \b{4}\def \c{-4}\def \r{11}\def \monicpol{x^{}+5}\def \longnbad{4x^{2}+16x^{}-9}\def \anspol{4x^{}-4}
\item {\bf (6 points)} 
 Divide the following using {\bf long division}. Your final answer should be in the form $$ \text{Quotient} + \dfrac{\text{Remainder}}{\text{Divisor}}.$$

\vspace{3mm}

$(\longnbad) \div (\monicpol)$

\vfill  \vfill \vfill
\newpage\def \discount{12}\def \paid{1177.14}\def \rainy{10.63}\def \orcost{1337.66}\def \purcost{1051.02}\def \orrainy{12.08}
\item {\bf (5 points)} 
 In April of this year, Greenfield received \rainy\ inches of rain. This was \discount\% less than the amount recorded in April of 2010. How much rain did Greenfield  receive in April 2010? Set up an algebraic equation to represent the situation and solve. Show units.

\vfill 
\def \insvar{24}\def \d{80}\def \zerospeed{43.82}\def \slimit{35}\def \s{50}\def \skidd{104.167}\def \safed{51.042}\def \rsafed{51}

 
\item Solve each and include units in your answer. Use the formula: $s = \sqrt{\insvar \cdot d}$ where $s$ is the speed of the car in miles per hour prior to braking, and $d$ is the stopping distance or length of the skid mark, in feet. 

\vspace{3mm}

Deone is driving down Bumpkin Road going $\slimit$ miles per hour when a fawn suddenly appears $\d$ feet away in the middle of the road. \begin{enumerate}
\item {\bf (5 points)} If she slams on her brakes now, how far will her car skid? \vspace{4cm}
\item {\bf (2 points)} Will she avoid hitting the fawn if it freezes in place? Why or why not? Fully explain your reason. \vspace{3cm}
\end{enumerate}


\newpage\def \xis{4}\def \yis{-1}\def \nomatcho{[5,2,2,3]}\def \a{5}\def \c{-2}\def \b{-2}\def \d{-3}\def \polyonesol{22}\def \polytwosol{-5}\def \xgoodone{5x^{}}\def \ygoodone{-2y^{}}\def \xgoodtwo{-2x^{}}\def \ygoodtwo{-3y^{}}\def \unitize{[0,0,0,1,0,1]}\def \mtem{3}\def \ntem{-5}\def \ptem{4}\def \qtem{3}\def \m{3}\def \n{-5}\def \p{4}\def \q{1}\def \polytonesol{17}\def \polyttwosol{15}\def \xtgoodone{3x^{}}\def \ytgoodone{-5y^{}}\def \xtgoodtwo{4x^{}}\def \ytgoodtwo{+y^{}}
\item {\bf (6 points)} 
 Solve the system using either substitution or elimination. Write your answer as an ordered pair, if possible. 

$ \begin{cases} \xtgoodone \ytgoodone = \polytonesol \\
\xtgoodtwo \ytgoodtwo = \polyttwosol
\end{cases}$

 \vfill \vfill
\newpage\def \vshift{1}\def \hshift{0}\def \chang{-2}\def \findval{-2}\def \yval{5}

 
\item Given the graph of $f(x)$ below, determine the following. {\bf Assume endpoints are included.}
\vspace{2mm}

\begin{tikzpicture}[scale=0.35]
	\def\startx{-10}
	\def\endx{10}
	\def\starty{-10}
	\def\endy{10}
	
	\draw [very thin,step=1,dotted] (\startx-.4, \starty-.4) grid (\endx+.4, \endy+.4);
	\draw[<->, thick] (\startx-.6,0) -- (\endx+.6, 0);
	\draw[<->, thick] (0,\starty-.6) -- (0,\endy+.6);
	\foreach \x in {\startx,...,\endx}
  	\draw[anchor=north] (\x-0.2, 0) node {\tiny $\x$};
	\foreach \y in {\starty,...,-2,-1,1,2,...,\endy}
  	\draw[anchor=east] (0, \y-.2) node {\tiny $\y$};
  	\draw (0.5, \endy+.3) node {$y$};
  	\draw (\endx+.5, 0.3) node {$x$};
	
	\draw (-4+\hshift,\vshift) node[fill,circle,scale=0.35]{} ;
	\draw (4+\hshift,-3+\vshift) node[fill,circle,scale=0.35]{};
	
	\draw[-, samples=100, very thick, domain=-4+\hshift:-2+\hshift]
	plot(\x, {2*(\x-\hshift)+8+\vshift});
	
	\draw[-,samples=100, very thick, domain=-2+\hshift:2+\hshift]
	plot(\x, {(-2)*(\x-\hshift)+\vshift});
	
	\draw[-, samples=100, very thick, domain=2+\hshift:3+\hshift]
	plot(\x, {(-4)+\vshift});
	
	\draw[-, samples=100, very thick, domain=3+\hshift:4+\hshift]
	plot(\x, {(\x-\hshift)-7+\vshift});
\end{tikzpicture} \begin{enumerate}

\item {\bf (2 points)} $f(\findval)$ \vspace{2cm}

\item {\bf (2 points)} The domain. \vspace{2cm}

\item {\bf (2 points)} The range. \vspace{2cm}

\end{enumerate}


\def \radi{4}\def \circumf{16}

 
\item Suppose the following circle has radius $r= \radi$. What is the circumference of the circle?
\vspace{2mm}

\includegraphics[scale = 0.8]{circle}

\vspace{1cm}
\newpage  $ $   \newpage\end{enumerate}\rhead{10002}\lhead{Sample Course}\chead{Sample Assessment - Fall 2023}\graphicspath{{C:/Users/iainc/anaconda3/Randomizer/Sample Course/Sample Assessment/}}\pagenumbering{arabic}\setcounter{page}{1}


\thispagestyle{fancy}

 \noindent Name: Ramanujan, Srinivasa \vspace{.3cm} \\\noindent Student ID: 8675310 \vspace{.3cm} \\\noindent Instructor: J. Brennan \vspace{.3cm} \\\noindent Signature: $\rule{6cm}{0.15mm}$ \vspace{.3cm} \\ 



\vspace{.4cm}

\noindent {\bf Note that the 'assessmentpreface.tex' file in the exams archive folder is read and placed here. This is also where student information is included, either to be replaced with information from the master.csv file or as blanks.}

\vspace{.4cm}

\hrule

\subsection*{Instructions:} \begin{enumerate}[1.]
\item Any cover page materials, per your departmental standards.
\end{enumerate}


\newpage
\begin{enumerate}
\item {\bf (2 points)} 
 Which of these is correct?

\begin{minipage}[t]{1.0\linewidth}\begin{multicols}{3}\begin{itemize}\item[(a)]  Wrong. \item[(b)]  Correct. \item[(c)]  Wrong. \end{itemize}\end{multicols}\end{minipage} \vfill 
\item {\bf (2 points)} 
 Which of these isn't mentally problematic?

\begin{minipage}[t]{1.0\linewidth}\begin{itemize}\item[(a)]  None of the below.  \item[(b)]  $a \neq a$ \item[(c)]  I've built a set that contains itself. \item[(d)]   All of the above. \end{itemize}\end{minipage} \vfill 
\item {\bf (2 points)} 
 Which of these is correct?

\begin{minipage}[t]{1.0\linewidth}\begin{multicols}{4}\begin{itemize}\item[(a)]  Wrong. \item[(e)]  Wrong. \item[(i)]  Wrong. \item[(b)]  Wrong. \item[(f)]  Wrong. \item[(j)]  Wrong. \item[(c)]  Wrong. \item[(g)]  Correct. \item[] \item[(d)]  Wrong. \item[(h)]  Wrong. \item[] \end{itemize}\end{multicols}\end{minipage} \vfill 
\item {\bf (2 points)} 
 The least common denominator for $\displaystyle \frac{x}{x+1}$ and $\displaystyle \frac{1}{x-1}$ is \vspace{.2cm}

\begin{minipage}[t]{1.0\linewidth}\begin{multicols}{2}\begin{itemize}\item[(a)]  $x+1$ \item[(c)]  $x(x+1)(x-1)$ \item[(b)]  $(x+1)(x-1)$ \item[(d)]  $x(x+1)$ \end{itemize}\end{multicols}\end{minipage} \vfill \newpage\def \a{7}\def \atwoone{1}\def \atwotwo{-6}\def \atwothree{3}\def \btwothree{6}\def \sumtwothree{9}\def \diftwothree{-3}\def \bigtwothree{300}\def \powtwothree{216}\def \logtwothree{0.6131471927654585}\def \factortwothree{26}\def \atwofour{1.02}\def \btwofour{1.204}\def \tooshorttwofour{10.1}\def \moneytwofour{10.10}\def \longertwofour{10.10000}\def \atwofive{0.12}\def \btwofive{0.12346}\def \athreeone{5}\def \bthreeone{7}\def \setthreetwo{[12, 6, 9]}\def \athreetwo{12}\def \bthreetwo{6}\def \cthreetwo{9}\def \controlthreethree{-5}\def \athreethree{3}\def \topthreethree{1}\def \athreefour{4}\def \bthreefour{1}\def \listthreefour{[1, 2, 3, 5]}\def \afourone{16}\def \bfourone{-2}\def \fracfourone{-8}\def \rootfourtwo{12}\def \simplifiedfourtwo{2 \sqrt{3}}\def \sqrtlistfourtwo{[2, 3]}\def \outfourtwo{2}\def \infourtwo{3}\def \wowfourtwo{1}\def \afourthree{0}\def \nicethreefour{3x^{2}-x^{}}\def \nastythreefour{xyz^{3}}\def \cfourthree{-4}\def \dfourthree{10}\def \infourthree{-4x^{}}\def \outfourthree{+10y^{}}\def \afourfour{1320774}\def \nicefourfour{1,320,774}\def \goodfourfour{1,000,000.12345}\def \badfourfour{1,000,000.1}
\item {\bf (12 points)} 
 $a = \a$ 
\vfill 
\def \a{7}\def \atwoone{1}\def \atwotwo{-6}\def \atwothree{2}\def \btwothree{9}\def \sumtwothree{11}\def \diftwothree{-7}\def \bigtwothree{200}\def \powtwothree{81}\def \logtwothree{0.3154648767857287}\def \factortwothree{38}\def \atwofour{1.94}\def \btwofour{1.322}\def \tooshorttwofour{10.1}\def \moneytwofour{10.10}\def \longertwofour{10.10000}\def \atwofive{0.12}\def \btwofive{0.12346}\def \athreeone{5}\def \bthreeone{3}\def \setthreetwo{[3, 7, 7]}\def \athreetwo{3}\def \bthreetwo{7}\def \cthreetwo{7}\def \controlthreethree{-4}\def \athreethree{1}\def \topthreethree{0}\def \athreefour{5}\def \bthreefour{3}\def \listthreefour{[1, 2, 3, 4]}\def \afourone{16}\def \bfourone{4}\def \fracfourone{4}\def \rootfourtwo{20}\def \simplifiedfourtwo{2 \sqrt{5}}\def \sqrtlistfourtwo{[2, 5]}\def \outfourtwo{2}\def \infourtwo{5}\def \wowfourtwo{1}\def \afourthree{5}\def \nicethreefour{3x^{2}-x^{}+5}\def \nastythreefour{xyz^{3}+5}\def \cfourthree{-4}\def \dfourthree{-9}\def \infourthree{-4x^{}}\def \outfourthree{-9y^{}}\def \afourfour{1872648}\def \nicefourfour{1,872,648}\def \goodfourfour{1,000,000.12345}\def \badfourfour{1,000,000.1}
\item {\bf (12 points)} 
 $\setthreetwo$ contains $\athreetwo,\bthreetwo$, and $\cthreetwo$. 
\vfill 
\def \a{7}\def \atwoone{1}\def \atwotwo{-2}\def \atwothree{1}\def \btwothree{6}\def \sumtwothree{7}\def \diftwothree{-5}\def \bigtwothree{100}\def \powtwothree{6}\def \logtwothree{0.0}\def \factortwothree{65}\def \atwofour{1.85}\def \btwofour{1.564}\def \tooshorttwofour{10.1}\def \moneytwofour{10.10}\def \longertwofour{10.10000}\def \atwofive{0.12}\def \btwofive{0.12346}\def \athreeone{5}\def \bthreeone{3}\def \setthreetwo{[12, 6, 9]}\def \athreetwo{12}\def \bthreetwo{6}\def \cthreetwo{9}\def \controlthreethree{-5}\def \athreethree{2}\def \topthreethree{0}\def \athreefour{5}\def \bthreefour{3}\def \listthreefour{[1, 2, 3, 4]}\def \afourone{12}\def \bfourone{-8}\def \fracfourone{\frac{-3}{2}}\def \rootfourtwo{8}\def \simplifiedfourtwo{2 \sqrt{2}}\def \sqrtlistfourtwo{[2, 2]}\def \outfourtwo{2}\def \infourtwo{2}\def \wowfourtwo{1}\def \afourthree{0}\def \nicethreefour{3x^{2}-x^{}}\def \nastythreefour{xyz^{3}}\def \cfourthree{-4}\def \dfourthree{9}\def \infourthree{-4x^{}}\def \outfourthree{+9y^{}}\def \afourfour{1235836}\def \nicefourfour{1,235,836}\def \goodfourfour{1,000,000.12345}\def \badfourfour{1,000,000.1}
\item {\bf (12 points)} 
 I successfully chose then number $\atwotwo$ at random. 
\vfill 
\newpage\def \x{68}\def \y{136}\def \L{272}\def \area{9248}
\item {\bf (5 points)} 
 You have $\L$ feet of fencing to enclose a rectangular plot that borders on a river. If you do not fence along the side of the river, find the \textbf{dimensions} of the plot that will maximize the area. \\

\begin{tikzpicture}[scale=0.2]
    \draw[snake=coil,segment aspect=0,gray] (0,0)   -- (20,0);
    \draw[black] (4,0) -- (4,-4) -- (16,-4) -- (16,0);
    \draw[gray] (10,1.5) node {River};
    \draw[black] (16,-4) node[anchor=west] {Fence};
\end{tikzpicture}
  
\vfill \vfill \vfill
\def \a{6}\def \b{4}\def \c{-7}\def \r{11}\def \monicpol{x^{}+6}\def \longnbad{4x^{2}+17x^{}-31}\def \anspol{4x^{}-7}
\item {\bf (6 points)} 
 Divide the following using {\bf long division}. Your final answer should be in the form $$ \text{Quotient} + \dfrac{\text{Remainder}}{\text{Divisor}}.$$

\vspace{3mm}

$(\longnbad) \div (\monicpol)$

\vfill  \vfill \vfill
\newpage\def \discount{15}\def \paid{1393.83}\def \rainy{12.74}\def \orcost{1639.80}\def \purcost{1212.03}\def \orrainy{14.99}
\item {\bf (5 points)} 
 Elridge Furniture discounts furniture \discount\% to customers paying cash. Jennifer paid \$\paid\ cash for a roll-top desk. What was the original price of the desk? (Round to the nearest cent.) Set up an algebraic equation to represent the situation and solve. Show units.

\vfill 
\def \insvar{27}\def \d{65}\def \zerospeed{41.89}\def \slimit{30}\def \s{55}\def \skidd{112.037}\def \safed{33.333}\def \rsafed{33}

 
\item Solve each and include units in your answer. Use the formula: $s = \sqrt{\insvar \cdot d}$ where $s$ is the speed of the car in miles per hour prior to braking, and $d$ is the stopping distance or length of the skid mark, in feet. 

\vspace{3mm}

Deone is driving down Bumpkin Road going $\slimit$ miles per hour when a fawn suddenly appears $\d$ feet away in the middle of the road. \begin{enumerate}
\item {\bf (5 points)} If she slams on her brakes now, how far will her car skid? \vspace{4cm}
\item {\bf (2 points)} Will she avoid hitting the fawn if it freezes in place? Why or why not? Fully explain your reason. \vspace{3cm}
\end{enumerate}


\newpage\def \xis{-3}\def \yis{-2}\def \nomatcho{[5,2,2,3]}\def \a{5}\def \c{-2}\def \b{-2}\def \d{-3}\def \polyonesol{-11}\def \polytwosol{12}\def \xgoodone{5x^{}}\def \ygoodone{-2y^{}}\def \xgoodtwo{-2x^{}}\def \ygoodtwo{-3y^{}}\def \unitize{[1,0,0,0,1,0]}\def \mtem{3}\def \ntem{4}\def \ptem{-5}\def \qtem{5}\def \m{1}\def \n{4}\def \p{-5}\def \q{5}\def \polytonesol{-11}\def \polyttwosol{5}\def \xtgoodone{x^{}}\def \ytgoodone{+4y^{}}\def \xtgoodtwo{-5x^{}}\def \ytgoodtwo{+5y^{}}
\item {\bf (6 points)} 
 Solve the system using either substitution or elimination. Write your answer as an ordered pair, if possible. 

$ \begin{cases} \xtgoodone \ytgoodone = \polytonesol \\
\xtgoodtwo \ytgoodtwo = \polyttwosol
\end{cases}$

 \vfill \vfill
\newpage\def \vshift{-5}\def \hshift{-4}\def \chang{0}\def \findval{-4}\def \yval{-5}

 
\item Given the graph of $f(x)$ below, determine the following. {\bf Assume endpoints are included.}
\vspace{2mm}

\begin{tikzpicture}[scale=0.35]
	\def\startx{-10}
	\def\endx{10}
	\def\starty{-10}
	\def\endy{10}
	
	\draw [very thin,step=1,dotted] (\startx-.4, \starty-.4) grid (\endx+.4, \endy+.4);
	\draw[<->, thick] (\startx-.6,0) -- (\endx+.6, 0);
	\draw[<->, thick] (0,\starty-.6) -- (0,\endy+.6);
	\foreach \x in {\startx,...,\endx}
  	\draw[anchor=north] (\x-0.2, 0) node {\tiny $\x$};
	\foreach \y in {\starty,...,-2,-1,1,2,...,\endy}
  	\draw[anchor=east] (0, \y-.2) node {\tiny $\y$};
  	\draw (0.5, \endy+.3) node {$y$};
  	\draw (\endx+.5, 0.3) node {$x$};
	
	\draw (-4+\hshift,\vshift) node[fill,circle,scale=0.35]{} ;
	\draw (4+\hshift,-3+\vshift) node[fill,circle,scale=0.35]{};
	
	\draw[-, samples=100, very thick, domain=-4+\hshift:-2+\hshift]
	plot(\x, {2*(\x-\hshift)+8+\vshift});
	
	\draw[-,samples=100, very thick, domain=-2+\hshift:2+\hshift]
	plot(\x, {(-2)*(\x-\hshift)+\vshift});
	
	\draw[-, samples=100, very thick, domain=2+\hshift:3+\hshift]
	plot(\x, {(-4)+\vshift});
	
	\draw[-, samples=100, very thick, domain=3+\hshift:4+\hshift]
	plot(\x, {(\x-\hshift)-7+\vshift});
\end{tikzpicture} \begin{enumerate}

\item {\bf (2 points)} $f(\findval)$ \vspace{2cm}

\item {\bf (2 points)} The domain. \vspace{2cm}

\item {\bf (2 points)} The range. \vspace{2cm}

\end{enumerate}


\def \radi{3}\def \circumf{9}

 
\item Suppose the following circle has radius $r= \radi$. What is the circumference of the circle?
\vspace{2mm}

\includegraphics[scale = 0.8]{circle}

\vspace{1cm}
\newpage  $ $   \newpage\end{enumerate}\rhead{10003}\lhead{Sample Course}\chead{Sample Assessment - Fall 2023}\graphicspath{{C:/Users/iainc/anaconda3/Randomizer/Sample Course/Sample Assessment/}}\pagenumbering{arabic}\setcounter{page}{1}


\thispagestyle{fancy}

 \noindent Name: Turing, Alan \vspace{.3cm} \\\noindent Student ID: 8675311 \vspace{.3cm} \\\noindent Instructor: I. Crump \vspace{.3cm} \\\noindent Signature: $\rule{6cm}{0.15mm}$ \vspace{.3cm} \\ 



\vspace{.4cm}

\noindent {\bf Note that the 'assessmentpreface.tex' file in the exams archive folder is read and placed here. This is also where student information is included, either to be replaced with information from the master.csv file or as blanks.}

\vspace{.4cm}

\hrule

\subsection*{Instructions:} \begin{enumerate}[1.]
\item Any cover page materials, per your departmental standards.
\end{enumerate}


\newpage
\begin{enumerate}
\item {\bf (2 points)} 
 Which of these is correct?

\begin{minipage}[t]{1.0\linewidth}\begin{multicols}{4}\begin{itemize}\item[(a)]  Wrong. \item[(e)]  Correct. \item[(i)]  Wrong. \item[(b)]  Wrong. \item[(f)]  Wrong. \item[(j)]  Wrong. \item[(c)]  Wrong. \item[(g)]  Wrong. \item[] \item[(d)]  Wrong. \item[(h)]  Wrong. \item[] \end{itemize}\end{multicols}\end{minipage} \vfill 
\item {\bf (2 points)} 
 Which of these isn't mentally problematic?

\begin{minipage}[t]{1.0\linewidth}\begin{itemize}\item[(a)]  None of the below.  \item[(b)]  I've built a set that contains itself. \item[(c)]  $a \neq a$ \item[(d)]   All of the above. \end{itemize}\end{minipage} \vfill 
\item {\bf (2 points)} 
 Which of these is correct?

\begin{minipage}[t]{1.0\linewidth}\begin{multicols}{4}\begin{itemize}\item[(a)]  Wrong. \item[(e)]  Wrong. \item[(i)]  Wrong. \item[(b)]  Wrong. \item[(f)]  Wrong. \item[(j)]  Correct. \item[(c)]  Wrong. \item[(g)]  Wrong. \item[] \item[(d)]  Wrong. \item[(h)]  Wrong. \item[] \end{itemize}\end{multicols}\end{minipage} \vfill 
\item {\bf (2 points)} 
 The least common denominator for $\displaystyle \frac{x}{x+1}$ and $\displaystyle \frac{1}{x-1}$ is \vspace{.2cm}

\begin{minipage}[t]{1.0\linewidth}\begin{multicols}{2}\begin{itemize}\item[(a)]  $x+1$ \item[(c)]  $(x+1)(x-1)$ \item[(b)]  $x(x+1)$ \item[(d)]  $x(x+1)(x-1)$ \end{itemize}\end{multicols}\end{minipage} \vfill \newpage\def \a{7}\def \atwoone{1}\def \atwotwo{-6}\def \atwothree{5}\def \btwothree{6}\def \sumtwothree{11}\def \diftwothree{-1}\def \bigtwothree{500}\def \powtwothree{7776}\def \logtwothree{0.8982444017039272}\def \factortwothree{187}\def \atwofour{1.2}\def \btwofour{1.95}\def \tooshorttwofour{10.1}\def \moneytwofour{10.10}\def \longertwofour{10.10000}\def \atwofive{0.12}\def \btwofive{0.12346}\def \athreeone{4}\def \bthreeone{2}\def \setthreetwo{[2, 5, 6]}\def \athreetwo{2}\def \bthreetwo{5}\def \cthreetwo{6}\def \controlthreethree{8}\def \athreethree{3}\def \topthreethree{0}\def \athreefour{3}\def \bthreefour{5}\def \listthreefour{[1, 2, 4, 5]}\def \afourone{16}\def \bfourone{-6}\def \fracfourone{\frac{-8}{3}}\def \rootfourtwo{20}\def \simplifiedfourtwo{2 \sqrt{5}}\def \sqrtlistfourtwo{[2, 5]}\def \outfourtwo{2}\def \infourtwo{5}\def \wowfourtwo{1}\def \afourthree{5}\def \nicethreefour{3x^{2}-x^{}+5}\def \nastythreefour{xyz^{3}+5}\def \cfourthree{-4}\def \dfourthree{9}\def \infourthree{-4x^{}}\def \outfourthree{+9y^{}}\def \afourfour{1416069}\def \nicefourfour{1,416,069}\def \goodfourfour{1,000,000.12345}\def \badfourfour{1,000,000.1}
\item {\bf (12 points)} 
 Any preamble. \begin{enumerate}
\item A first part. \vspace{2cm}
\item A second part.
\end{enumerate}

\vfill 
\def \a{7}\def \atwoone{1}\def \atwotwo{-2}\def \atwothree{5}\def \btwothree{8}\def \sumtwothree{13}\def \diftwothree{-3}\def \bigtwothree{500}\def \powtwothree{32768}\def \logtwothree{0.7739760316291208}\def \factortwothree{85}\def \atwofour{1.12}\def \btwofour{1.114}\def \tooshorttwofour{10.1}\def \moneytwofour{10.10}\def \longertwofour{10.10000}\def \atwofive{0.12}\def \btwofive{0.12346}\def \athreeone{5}\def \bthreeone{6}\def \setthreetwo{[12, 6, 9]}\def \athreetwo{12}\def \bthreetwo{6}\def \cthreetwo{9}\def \controlthreethree{-8}\def \athreethree{3}\def \topthreethree{1}\def \athreefour{3}\def \bthreefour{5}\def \listthreefour{[1, 2, 4, 5]}\def \afourone{16}\def \bfourone{4}\def \fracfourone{4}\def \rootfourtwo{12}\def \simplifiedfourtwo{2 \sqrt{3}}\def \sqrtlistfourtwo{[2, 3]}\def \outfourtwo{2}\def \infourtwo{3}\def \wowfourtwo{1}\def \afourthree{0}\def \nicethreefour{3x^{2}-x^{}}\def \nastythreefour{xyz^{3}}\def \cfourthree{4}\def \dfourthree{-10}\def \infourthree{4x^{}}\def \outfourthree{-10y^{}}\def \afourfour{1274076}\def \nicefourfour{1,274,076}\def \goodfourfour{1,000,000.12345}\def \badfourfour{1,000,000.1}
\item {\bf (12 points)} 
 $\dfrac{1+\sqrt{\rootfourtwo}}{2} = \dfrac{1}{2} + \ifthenelse{\wowfourtwo = 1}{}{\wowfourtwo} \sqrt{\infourtwo}$

\vfill 
\def \a{7}\def \atwoone{1}\def \atwotwo{3}\def \atwothree{5}\def \btwothree{7}\def \sumtwothree{12}\def \diftwothree{-2}\def \bigtwothree{500}\def \powtwothree{16807}\def \logtwothree{0.8270874753469162}\def \factortwothree{187}\def \atwofour{1.69}\def \btwofour{1.099}\def \tooshorttwofour{10.1}\def \moneytwofour{10.10}\def \longertwofour{10.10000}\def \atwofive{0.12}\def \btwofive{0.12346}\def \athreeone{5}\def \bthreeone{6}\def \setthreetwo{[3, 7, 7]}\def \athreetwo{3}\def \bthreetwo{7}\def \cthreetwo{7}\def \controlthreethree{4}\def \athreethree{1}\def \topthreethree{1}\def \athreefour{5}\def \bthreefour{4}\def \listthreefour{[1, 2, 3, 4]}\def \afourone{12}\def \bfourone{-2}\def \fracfourone{-6}\def \rootfourtwo{20}\def \simplifiedfourtwo{2 \sqrt{5}}\def \sqrtlistfourtwo{[2, 5]}\def \outfourtwo{2}\def \infourtwo{5}\def \wowfourtwo{1}\def \afourthree{0}\def \nicethreefour{3x^{2}-x^{}}\def \nastythreefour{xyz^{3}}\def \cfourthree{-4}\def \dfourthree{-10}\def \infourthree{-4x^{}}\def \outfourthree{-10y^{}}\def \afourfour{1456961}\def \nicefourfour{1,456,961}\def \goodfourfour{1,000,000.12345}\def \badfourfour{1,000,000.1}
\item {\bf (12 points)} 
 $\atwofour$, $\btwofour$ 
\vfill 
\newpage\def \x{61}\def \y{122}\def \L{244}\def \area{7442}
\item {\bf (5 points)} 
 You have $\L$ feet of fencing to enclose a rectangular plot that borders on a river. If you do not fence along the side of the river, what is the largest area that can be enclosed? \\

\begin{tikzpicture}[scale=0.2]
    \draw[snake=coil,segment aspect=0,gray] (0,0)   -- (20,0);
    \draw[black] (4,0) -- (4,-4) -- (16,-4) -- (16,0);
    \draw[gray] (10,1.5) node {River};
    \draw[black] (16,-4) node[anchor=west] {Fence};
\end{tikzpicture}
  
\vfill \vfill \vfill
\def \a{4}\def \b{2}\def \c{-3}\def \r{13}\def \monicpol{x^{}+4}\def \longnbad{2x^{2}+5x^{}+1}\def \anspol{2x^{}-3}
\item {\bf (6 points)} 
 Divide the following using {\bf long division}. Your final answer should be in the form $$ \text{Quotient} + \dfrac{\text{Remainder}}{\text{Divisor}}.$$

\vspace{3mm}

$(\longnbad) \div (\monicpol)$

\vfill  \vfill \vfill
\newpage\def \discount{17}\def \paid{1125.60}\def \rainy{9.70}\def \orcost{1356.14}\def \purcost{962.05}\def \orrainy{11.69}
\item {\bf (5 points)} 
 Elridge Furniture discounts furniture \discount\% to customers paying cash. Jennifer paid \$\paid\ cash for a roll-top desk. What was the original price of the desk? (Round to the nearest cent.) Set up an algebraic equation to represent the situation and solve. Show units.

\vfill 
\def \insvar{24}\def \d{80}\def \zerospeed{43.82}\def \slimit{35}\def \s{51}\def \skidd{108.375}\def \safed{51.042}\def \rsafed{51}

 
\item Solve each and include units in your answer. Use the formula: $s = \sqrt{\insvar \cdot d}$ where $s$ is the speed of the car in miles per hour prior to braking, and $d$ is the stopping distance or length of the skid mark, in feet. 

\vspace{3mm}

Deone is driving down Bumpkin Road going $\slimit$ miles per hour when a fawn suddenly appears $\d$ feet away in the middle of the road. \begin{enumerate}
\item {\bf (5 points)} If she slams on her brakes now, how far will her car skid? \vspace{4cm}
\item {\bf (2 points)} Will she avoid hitting the fawn if it freezes in place? Why or why not? Fully explain your reason. \vspace{3cm}
\end{enumerate}


\newpage\def \xis{-4}\def \yis{4}\def \nomatcho{[3,5,2,3]}\def \a{3}\def \c{5}\def \b{-2}\def \d{-3}\def \polyonesol{-20}\def \polytwosol{-32}\def \xgoodone{3x^{}}\def \ygoodone{-2y^{}}\def \xgoodtwo{5x^{}}\def \ygoodtwo{-3y^{}}\def \unitize{[0,0,1,0,0,1]}\def \mtem{2}\def \ntem{4}\def \ptem{-5}\def \qtem{5}\def \m{2}\def \n{4}\def \p{1}\def \q{5}\def \polytonesol{8}\def \polyttwosol{16}\def \xtgoodone{2x^{}}\def \ytgoodone{+4y^{}}\def \xtgoodtwo{x^{}}\def \ytgoodtwo{+5y^{}}
\item {\bf (6 points)} 
 Solve the system using either substitution or elimination. Write your answer as an ordered pair, if possible. 

$ \begin{cases} \xtgoodone \ytgoodone = \polytonesol \\
\xtgoodtwo \ytgoodtwo = \polyttwosol
\end{cases}$

 \vfill \vfill
\newpage\def \vshift{-1}\def \hshift{2}\def \chang{0}\def \findval{2}\def \yval{-1}

 
\item Given the graph of $f(x)$ below, determine the following. {\bf Assume endpoints are included.}
\vspace{2mm}

\begin{tikzpicture}[scale=0.35]
	\def\startx{-10}
	\def\endx{10}
	\def\starty{-10}
	\def\endy{10}
	
	\draw [very thin,step=1,dotted] (\startx-.4, \starty-.4) grid (\endx+.4, \endy+.4);
	\draw[<->, thick] (\startx-.6,0) -- (\endx+.6, 0);
	\draw[<->, thick] (0,\starty-.6) -- (0,\endy+.6);
	\foreach \x in {\startx,...,\endx}
  	\draw[anchor=north] (\x-0.2, 0) node {\tiny $\x$};
	\foreach \y in {\starty,...,-2,-1,1,2,...,\endy}
  	\draw[anchor=east] (0, \y-.2) node {\tiny $\y$};
  	\draw (0.5, \endy+.3) node {$y$};
  	\draw (\endx+.5, 0.3) node {$x$};
	
	\draw (-4+\hshift,\vshift) node[fill,circle,scale=0.35]{} ;
	\draw (4+\hshift,-3+\vshift) node[fill,circle,scale=0.35]{};
	
	\draw[-, samples=100, very thick, domain=-4+\hshift:-2+\hshift]
	plot(\x, {2*(\x-\hshift)+8+\vshift});
	
	\draw[-,samples=100, very thick, domain=-2+\hshift:2+\hshift]
	plot(\x, {(-2)*(\x-\hshift)+\vshift});
	
	\draw[-, samples=100, very thick, domain=2+\hshift:3+\hshift]
	plot(\x, {(-4)+\vshift});
	
	\draw[-, samples=100, very thick, domain=3+\hshift:4+\hshift]
	plot(\x, {(\x-\hshift)-7+\vshift});
\end{tikzpicture} \begin{enumerate}

\item {\bf (2 points)} $f(\findval)$ \vspace{2cm}

\item {\bf (2 points)} The domain. \vspace{2cm}

\item {\bf (2 points)} The range. \vspace{2cm}

\end{enumerate}


\def \radi{5}\def \circumf{25}

 
\item Suppose the following circle has radius $r= \radi$. What is the circumference of the circle?
\vspace{2mm}

\includegraphics[scale = 0.8]{circle}

\vspace{1cm}
\newpage  $ $   \newpage\end{enumerate}\rhead{10004}\lhead{Sample Course}\chead{Sample Assessment - Fall 2023}\graphicspath{{C:/Users/iainc/anaconda3/Randomizer/Sample Course/Sample Assessment/}}\pagenumbering{arabic}\setcounter{page}{1}


\thispagestyle{fancy}

 \noindent Name: Von Neumann, John \vspace{.3cm} \\\noindent Student ID: 8675312 \vspace{.3cm} \\\noindent Instructor: J. Niknejad \vspace{.3cm} \\\noindent Signature: $\rule{6cm}{0.15mm}$ \vspace{.3cm} \\ 



\vspace{.4cm}

\noindent {\bf Note that the 'assessmentpreface.tex' file in the exams archive folder is read and placed here. This is also where student information is included, either to be replaced with information from the master.csv file or as blanks.}

\vspace{.4cm}

\hrule

\subsection*{Instructions:} \begin{enumerate}[1.]
\item Any cover page materials, per your departmental standards.
\end{enumerate}


\newpage
\begin{enumerate}
\item {\bf (2 points)} 
 Which of these isn't mentally problematic?

\begin{minipage}[t]{1.0\linewidth}\begin{itemize}\item[(a)]  None of the below.  \item[(b)]  I've built a set that contains itself. \item[(c)]  $a \neq a$ \item[(d)]   All of the above. \end{itemize}\end{minipage} \vfill 
\item {\bf (2 points)} 
 Which of these is correct?

\begin{minipage}[t]{1.0\linewidth}\begin{multicols}{3}\begin{itemize}\item[(a)]  Wrong. \item[(b)]  Correct. \item[(c)]  Wrong. \end{itemize}\end{multicols}\end{minipage} \vfill 
\item {\bf (2 points)} 
 Which of these is correct?

\begin{minipage}[t]{1.0\linewidth}\begin{multicols}{4}\begin{itemize}\item[(a)]  Wrong. \item[(e)]  Wrong. \item[(i)]  Wrong. \item[(b)]  Wrong. \item[(f)]  Wrong. \item[(j)]  Correct. \item[(c)]  Wrong. \item[(g)]  Wrong. \item[] \item[(d)]  Wrong. \item[(h)]  Wrong. \item[] \end{itemize}\end{multicols}\end{minipage} \vfill 
\item {\bf (2 points)} 
 The least common denominator for $\displaystyle \frac{x}{x+1}$ and $\displaystyle \frac{1}{x(x-1)}$ is \vspace{.2cm}

\begin{minipage}[t]{1.0\linewidth}\begin{multicols}{2}\begin{itemize}\item[(a)]  $x(x+1)(x-1)$ \item[(c)]  $(x+1)(x-1)$ \item[(b)]  $x+1$ \item[(d)]  $x(x+1)$ \end{itemize}\end{multicols}\end{minipage} \vfill \newpage\def \a{7}\def \atwoone{2}\def \atwotwo{-2}\def \atwothree{5}\def \btwothree{6}\def \sumtwothree{11}\def \diftwothree{-1}\def \bigtwothree{500}\def \powtwothree{7776}\def \logtwothree{0.8982444017039272}\def \factortwothree{119}\def \atwofour{1.15}\def \btwofour{1.96}\def \tooshorttwofour{10.1}\def \moneytwofour{10.10}\def \longertwofour{10.10000}\def \atwofive{0.12}\def \btwofive{0.12346}\def \athreeone{6}\def \bthreeone{8}\def \setthreetwo{[3, 7, 7]}\def \athreetwo{3}\def \bthreetwo{7}\def \cthreetwo{7}\def \controlthreethree{5}\def \athreethree{1}\def \topthreethree{0}\def \athreefour{5}\def \bthreefour{3}\def \listthreefour{[1, 2, 3, 4]}\def \afourone{8}\def \bfourone{4}\def \fracfourone{2}\def \rootfourtwo{20}\def \simplifiedfourtwo{2 \sqrt{5}}\def \sqrtlistfourtwo{[2, 5]}\def \outfourtwo{2}\def \infourtwo{5}\def \wowfourtwo{1}\def \afourthree{0}\def \nicethreefour{3x^{2}-x^{}}\def \nastythreefour{xyz^{3}}\def \cfourthree{-4}\def \dfourthree{-9}\def \infourthree{-4x^{}}\def \outfourthree{-9y^{}}\def \afourfour{1596068}\def \nicefourfour{1,596,068}\def \goodfourfour{1,000,000.12345}\def \badfourfour{1,000,000.1}
\item {\bf (12 points)} 
 $\afourfour$ is awkward to read; \nicefourfour is missing a space, \nicefourfour\ is nice, $\nicefourfour$ adds spacing that makes it confusing to read. 
\vfill 
\def \a{7}\def \atwoone{3}\def \atwotwo{5}\def \atwothree{4}\def \btwothree{8}\def \sumtwothree{12}\def \diftwothree{-4}\def \bigtwothree{400}\def \powtwothree{4096}\def \logtwothree{0.6666666666666667}\def \factortwothree{39}\def \atwofour{1.19}\def \btwofour{1.487}\def \tooshorttwofour{10.1}\def \moneytwofour{10.10}\def \longertwofour{10.10000}\def \atwofive{0.12}\def \btwofive{0.12346}\def \athreeone{4}\def \bthreeone{6}\def \setthreetwo{[3, 7, 7]}\def \athreetwo{3}\def \bthreetwo{7}\def \cthreetwo{7}\def \controlthreethree{8}\def \athreethree{3}\def \topthreethree{1}\def \athreefour{5}\def \bthreefour{3}\def \listthreefour{[1, 2, 3, 4]}\def \afourone{16}\def \bfourone{-2}\def \fracfourone{-8}\def \rootfourtwo{12}\def \simplifiedfourtwo{2 \sqrt{3}}\def \sqrtlistfourtwo{[2, 3]}\def \outfourtwo{2}\def \infourtwo{3}\def \wowfourtwo{1}\def \afourthree{5}\def \nicethreefour{3x^{2}-x^{}+5}\def \nastythreefour{xyz^{3}+5}\def \cfourthree{4}\def \dfourthree{-9}\def \infourthree{4x^{}}\def \outfourthree{-9y^{}}\def \afourfour{1729798}\def \nicefourfour{1,729,798}\def \goodfourfour{1,000,000.12345}\def \badfourfour{1,000,000.1}
\item {\bf (12 points)} 
 I successfully chose then number $\atwotwo$ at random. 
\vfill 
\def \a{7}\def \atwoone{3}\def \atwotwo{-2}\def \atwothree{1}\def \btwothree{6}\def \sumtwothree{7}\def \diftwothree{-5}\def \bigtwothree{100}\def \powtwothree{6}\def \logtwothree{0.0}\def \factortwothree{85}\def \atwofour{1.56}\def \btwofour{1.532}\def \tooshorttwofour{10.1}\def \moneytwofour{10.10}\def \longertwofour{10.10000}\def \atwofive{0.12}\def \btwofive{0.12346}\def \athreeone{5}\def \bthreeone{7}\def \setthreetwo{[12, 6, 9]}\def \athreetwo{12}\def \bthreetwo{6}\def \cthreetwo{9}\def \controlthreethree{5}\def \athreethree{3}\def \topthreethree{1}\def \athreefour{4}\def \bthreefour{2}\def \listthreefour{[1, 2, 3, 5]}\def \afourone{8}\def \bfourone{-8}\def \fracfourone{-1}\def \rootfourtwo{8}\def \simplifiedfourtwo{2 \sqrt{2}}\def \sqrtlistfourtwo{[2, 2]}\def \outfourtwo{2}\def \infourtwo{2}\def \wowfourtwo{1}\def \afourthree{0}\def \nicethreefour{3x^{2}-x^{}}\def \nastythreefour{xyz^{3}}\def \cfourthree{-4}\def \dfourthree{-9}\def \infourthree{-4x^{}}\def \outfourthree{-9y^{}}\def \afourfour{1848409}\def \nicefourfour{1,848,409}\def \goodfourfour{1,000,000.12345}\def \badfourfour{1,000,000.1}
\item {\bf (12 points)} 
 I successfully chose then number $\atwotwo$ at random. 
\vfill 
\newpage\def \x{65}\def \y{130}\def \L{260}\def \area{8450}
\item {\bf (5 points)} 
 You have $\L$ feet of fencing to enclose a rectangular plot that borders on a river. If you do not fence along the side of the river, what is the largest area that can be enclosed? \\

\begin{tikzpicture}[scale=0.2]
    \draw[snake=coil,segment aspect=0,gray] (0,0)   -- (20,0);
    \draw[black] (4,0) -- (4,-4) -- (16,-4) -- (16,0);
    \draw[gray] (10,1.5) node {River};
    \draw[black] (16,-4) node[anchor=west] {Fence};
\end{tikzpicture}
  
\vfill \vfill \vfill
\def \a{4}\def \b{2}\def \c{-6}\def \r{11}\def \monicpol{x^{}+4}\def \longnbad{2x^{2}+2x^{}-13}\def \anspol{2x^{}-6}
\item {\bf (6 points)} 
 Divide the following using {\bf long division}. Your final answer should be in the form $$ \text{Quotient} + \dfrac{\text{Remainder}}{\text{Divisor}}.$$

\vspace{3mm}

$(\longnbad) \div (\monicpol)$

\vfill  \vfill \vfill
\newpage\def \discount{18}\def \paid{1443.14}\def \rainy{9.13}\def \orcost{1759.93}\def \purcost{1223.00}\def \orrainy{11.13}
\item {\bf (5 points)} 
 Elridge Furniture discounts furniture \discount\% to customers paying cash. Jennifer paid \$\paid\ cash for a roll-top desk. What was the original price of the desk? (Round to the nearest cent.) Set up an algebraic equation to represent the situation and solve. Show units.

\vfill 
\def \insvar{30}\def \d{60}\def \zerospeed{42.43}\def \slimit{35}\def \s{54}\def \skidd{97.2}\def \safed{40.833}\def \rsafed{40}

 
\item Solve each and include units in your answer. Use the formula: $s = \sqrt{\insvar \cdot d}$ where $s$ is the speed of the car in miles per hour prior to braking, and $d$ is the stopping distance or length of the skid mark, in feet. 

\vspace{3mm}

Deone is driving down Bumpkin Road going $\slimit$ miles per hour when a fawn suddenly appears $\d$ feet away in the middle of the road. \begin{enumerate}
\item {\bf (5 points)} If she slams on her brakes now, how far will her car skid? \vspace{4cm}
\item {\bf (2 points)} Will she avoid hitting the fawn if it freezes in place? Why or why not? Fully explain your reason. \vspace{3cm}
\end{enumerate}


\newpage\def \xis{1}\def \yis{4}\def \nomatcho{[2,5,3,2]}\def \a{2}\def \c{-5}\def \b{-3}\def \d{2}\def \polyonesol{-10}\def \polytwosol{3}\def \xgoodone{2x^{}}\def \ygoodone{-3y^{}}\def \xgoodtwo{-5x^{}}\def \ygoodtwo{+2y^{}}\def \unitize{[0,1,0,0,1,0]}\def \mtem{3}\def \ntem{4}\def \ptem{-4}\def \qtem{-5}\def \m{3}\def \n{1}\def \p{-4}\def \q{-5}\def \polytonesol{7}\def \polyttwosol{-24}\def \xtgoodone{3x^{}}\def \ytgoodone{+y^{}}\def \xtgoodtwo{-4x^{}}\def \ytgoodtwo{-5y^{}}
\item {\bf (6 points)} 
 Solve the system using either substitution or elimination. Write your answer as an ordered pair, if possible. 

$ \begin{cases} \xtgoodone \ytgoodone = \polytonesol \\
\xtgoodtwo \ytgoodtwo = \polyttwosol
\end{cases}$

 \vfill \vfill
\newpage\def \vshift{-3}\def \hshift{-4}\def \chang{-1}\def \findval{-5}\def \yval{-1}

 
\item Given the graph of $f(x)$ below, determine the following. {\bf Assume endpoints are included.}
\vspace{2mm}

\begin{tikzpicture}[scale=0.35]
	\def\startx{-10}
	\def\endx{10}
	\def\starty{-10}
	\def\endy{10}
	
	\draw [very thin,step=1,dotted] (\startx-.4, \starty-.4) grid (\endx+.4, \endy+.4);
	\draw[<->, thick] (\startx-.6,0) -- (\endx+.6, 0);
	\draw[<->, thick] (0,\starty-.6) -- (0,\endy+.6);
	\foreach \x in {\startx,...,\endx}
  	\draw[anchor=north] (\x-0.2, 0) node {\tiny $\x$};
	\foreach \y in {\starty,...,-2,-1,1,2,...,\endy}
  	\draw[anchor=east] (0, \y-.2) node {\tiny $\y$};
  	\draw (0.5, \endy+.3) node {$y$};
  	\draw (\endx+.5, 0.3) node {$x$};
	
	\draw (-4+\hshift,\vshift) node[fill,circle,scale=0.35]{} ;
	\draw (4+\hshift,-3+\vshift) node[fill,circle,scale=0.35]{};
	
	\draw[-, samples=100, very thick, domain=-4+\hshift:-2+\hshift]
	plot(\x, {2*(\x-\hshift)+8+\vshift});
	
	\draw[-,samples=100, very thick, domain=-2+\hshift:2+\hshift]
	plot(\x, {(-2)*(\x-\hshift)+\vshift});
	
	\draw[-, samples=100, very thick, domain=2+\hshift:3+\hshift]
	plot(\x, {(-4)+\vshift});
	
	\draw[-, samples=100, very thick, domain=3+\hshift:4+\hshift]
	plot(\x, {(\x-\hshift)-7+\vshift});
\end{tikzpicture} \begin{enumerate}

\item {\bf (2 points)} $f(\findval)$ \vspace{2cm}

\item {\bf (2 points)} The domain. \vspace{2cm}

\item {\bf (2 points)} The range. \vspace{2cm}

\end{enumerate}


\def \radi{2}\def \circumf{4}

 
\item Suppose the following circle has radius $r= \radi$. What is the circumference of the circle?
\vspace{2mm}

\includegraphics[scale = 0.8]{circle}

\vspace{1cm}
\newpage  $ $   \newpage\end{enumerate}\rhead{10005}\lhead{Sample Course}\chead{Sample Assessment - Fall 2023}\graphicspath{{C:/Users/iainc/anaconda3/Randomizer/Sample Course/Sample Assessment/}}\pagenumbering{arabic}\setcounter{page}{1}


\thispagestyle{fancy}

 \noindent Name: Euler, Leonhard \vspace{.3cm} \\\noindent Student ID: 8675313 \vspace{.3cm} \\\noindent Instructor: J. Niknejad \vspace{.3cm} \\\noindent Signature: $\rule{6cm}{0.15mm}$ \vspace{.3cm} \\ 



\vspace{.4cm}

\noindent {\bf Note that the 'assessmentpreface.tex' file in the exams archive folder is read and placed here. This is also where student information is included, either to be replaced with information from the master.csv file or as blanks.}

\vspace{.4cm}

\hrule

\subsection*{Instructions:} \begin{enumerate}[1.]
\item Any cover page materials, per your departmental standards.
\end{enumerate}


\newpage
\begin{enumerate}
\item {\bf (2 points)} 
 Which of these isn't mentally problematic?

\begin{minipage}[t]{1.0\linewidth}\begin{itemize}\item[(a)]  None of the below.  \item[(b)]  $a \neq a$ \item[(c)]  I've built a set that contains itself. \item[(d)]   All of the above. \end{itemize}\end{minipage} \vfill 
\item {\bf (2 points)} 
 Which of these is correct?

\begin{minipage}[t]{1.0\linewidth}\begin{multicols}{3}\begin{itemize}\item[(a)]  Correct. \item[(b)]  Wrong. \item[(c)]  Wrong. \end{itemize}\end{multicols}\end{minipage} \vfill 
\item {\bf (2 points)} 
 Which of these is correct?

\begin{minipage}[t]{1.0\linewidth}\begin{multicols}{3}\begin{itemize}\item[(a)]  Wrong. \item[(b)]  Wrong. \item[(c)]  Correct. \end{itemize}\end{multicols}\end{minipage} \vfill 
\item {\bf (2 points)} 
 The least common denominator for $\displaystyle \frac{x}{x+1}$ and $\displaystyle \frac{1}{x-1}$ is \vspace{.2cm}

\begin{minipage}[t]{1.0\linewidth}\begin{multicols}{2}\begin{itemize}\item[(a)]  $x+1$ \item[(c)]  $x(x+1)$ \item[(b)]  $x(x+1)(x-1)$ \item[(d)]  $(x+1)(x-1)$ \end{itemize}\end{multicols}\end{minipage} \vfill \newpage\def \a{7}\def \atwoone{1}\def \atwotwo{-2}\def \atwothree{4}\def \btwothree{7}\def \sumtwothree{11}\def \diftwothree{-3}\def \bigtwothree{400}\def \powtwothree{2401}\def \logtwothree{0.7124143742160444}\def \factortwothree{34}\def \atwofour{1.84}\def \btwofour{1.516}\def \tooshorttwofour{10.1}\def \moneytwofour{10.10}\def \longertwofour{10.10000}\def \atwofive{0.12}\def \btwofive{0.12346}\def \athreeone{6}\def \bthreeone{7}\def \setthreetwo{[3, 7, 7]}\def \athreetwo{3}\def \bthreetwo{7}\def \cthreetwo{7}\def \controlthreethree{4}\def \athreethree{4}\def \topthreethree{1}\def \athreefour{5}\def \bthreefour{3}\def \listthreefour{[1, 2, 3, 4]}\def \afourone{16}\def \bfourone{-2}\def \fracfourone{-8}\def \rootfourtwo{20}\def \simplifiedfourtwo{2 \sqrt{5}}\def \sqrtlistfourtwo{[2, 5]}\def \outfourtwo{2}\def \infourtwo{5}\def \wowfourtwo{1}\def \afourthree{5}\def \nicethreefour{3x^{2}-x^{}+5}\def \nastythreefour{xyz^{3}+5}\def \cfourthree{4}\def \dfourthree{-9}\def \infourthree{4x^{}}\def \outfourthree{-9y^{}}\def \afourfour{1189836}\def \nicefourfour{1,189,836}\def \goodfourfour{1,000,000.12345}\def \badfourfour{1,000,000.1}
\item {\bf (12 points)} 
  I successfully chose then number $\atwoone$ at random. 
\vfill 
\def \a{7}\def \atwoone{2}\def \atwotwo{-2}\def \atwothree{1}\def \btwothree{6}\def \sumtwothree{7}\def \diftwothree{-5}\def \bigtwothree{100}\def \powtwothree{6}\def \logtwothree{0.0}\def \factortwothree{119}\def \atwofour{1.2}\def \btwofour{1.725}\def \tooshorttwofour{10.1}\def \moneytwofour{10.10}\def \longertwofour{10.10000}\def \atwofive{0.12}\def \btwofive{0.12346}\def \athreeone{4}\def \bthreeone{6}\def \setthreetwo{[12, 6, 9]}\def \athreetwo{12}\def \bthreetwo{6}\def \cthreetwo{9}\def \controlthreethree{4}\def \athreethree{2}\def \topthreethree{1}\def \athreefour{4}\def \bthreefour{3}\def \listthreefour{[1, 2, 3, 5]}\def \afourone{4}\def \bfourone{4}\def \fracfourone{1}\def \rootfourtwo{20}\def \simplifiedfourtwo{2 \sqrt{5}}\def \sqrtlistfourtwo{[2, 5]}\def \outfourtwo{2}\def \infourtwo{5}\def \wowfourtwo{1}\def \afourthree{5}\def \nicethreefour{3x^{2}-x^{}+5}\def \nastythreefour{xyz^{3}+5}\def \cfourthree{-4}\def \dfourthree{9}\def \infourthree{-4x^{}}\def \outfourthree{+9y^{}}\def \afourfour{1299051}\def \nicefourfour{1,299,051}\def \goodfourfour{1,000,000.12345}\def \badfourfour{1,000,000.1}
\item {\bf (12 points)} 
 $\dfrac{1+\sqrt{\rootfourtwo}}{2} = \dfrac{1}{2} + \ifthenelse{\wowfourtwo = 1}{}{\wowfourtwo} \sqrt{\infourtwo}$

\vfill 
\def \a{7}\def \atwoone{3}\def \atwotwo{-6}\def \atwothree{5}\def \btwothree{8}\def \sumtwothree{13}\def \diftwothree{-3}\def \bigtwothree{500}\def \powtwothree{32768}\def \logtwothree{0.7739760316291208}\def \factortwothree{119}\def \atwofour{1.89}\def \btwofour{1.679}\def \tooshorttwofour{10.1}\def \moneytwofour{10.10}\def \longertwofour{10.10000}\def \atwofive{0.12}\def \btwofive{0.12346}\def \athreeone{6}\def \bthreeone{5}\def \setthreetwo{[2, 5, 6]}\def \athreetwo{2}\def \bthreetwo{5}\def \cthreetwo{6}\def \controlthreethree{-4}\def \athreethree{2}\def \topthreethree{1}\def \athreefour{3}\def \bthreefour{1}\def \listthreefour{[1, 2, 4, 5]}\def \afourone{16}\def \bfourone{4}\def \fracfourone{4}\def \rootfourtwo{8}\def \simplifiedfourtwo{2 \sqrt{2}}\def \sqrtlistfourtwo{[2, 2]}\def \outfourtwo{2}\def \infourtwo{2}\def \wowfourtwo{1}\def \afourthree{5}\def \nicethreefour{3x^{2}-x^{}+5}\def \nastythreefour{xyz^{3}+5}\def \cfourthree{-4}\def \dfourthree{-9}\def \infourthree{-4x^{}}\def \outfourthree{-9y^{}}\def \afourfour{1600731}\def \nicefourfour{1,600,731}\def \goodfourfour{1,000,000.12345}\def \badfourfour{1,000,000.1}
\item {\bf (12 points)} 
  $\infourthree \outfourthree$ 
\vfill 
\newpage\def \x{82}\def \y{164}\def \L{328}\def \area{13448}
\item {\bf (5 points)} 
 You have $\L$ feet of fencing to enclose a rectangular plot that borders on a river. If you do not fence along the side of the river, what is the largest area that can be enclosed? \\

\begin{tikzpicture}[scale=0.2]
    \draw[snake=coil,segment aspect=0,gray] (0,0)   -- (20,0);
    \draw[black] (4,0) -- (4,-4) -- (16,-4) -- (16,0);
    \draw[gray] (10,1.5) node {River};
    \draw[black] (16,-4) node[anchor=west] {Fence};
\end{tikzpicture}
  
\vfill \vfill \vfill
\def \a{4}\def \b{4}\def \c{-6}\def \r{10}\def \monicpol{x^{}+4}\def \longnbad{4x^{2}+10x^{}-14}\def \anspol{4x^{}-6}
\item {\bf (6 points)} 
 Divide the following using {\bf long division}. Your final answer should be in the form $$ \text{Quotient} + \dfrac{\text{Remainder}}{\text{Divisor}}.$$

\vspace{3mm}

$(\longnbad) \div (\monicpol)$

\vfill  \vfill \vfill
\newpage\def \discount{14}\def \paid{1239.31}\def \rainy{12.31}\def \orcost{1441.06}\def \purcost{1087.11}\def \orrainy{14.31}
\item {\bf (5 points)} 
 Elridge Furniture discounts furniture \discount\% to customers paying cash. Jennifer paid \$\paid\ cash for a roll-top desk. What was the original price of the desk? (Round to the nearest cent.) Set up an algebraic equation to represent the situation and solve. Show units.

\vfill 
\def \insvar{21}\def \d{55}\def \zerospeed{33.99}\def \slimit{45}\def \s{57}\def \skidd{154.714}\def \safed{96.429}\def \rsafed{96}

 
\item Solve each and include units in your answer. Use the formula: $s = \sqrt{\insvar \cdot d}$ where $s$ is the speed of the car in miles per hour prior to braking, and $d$ is the stopping distance or length of the skid mark, in feet. 

\vspace{3mm}

Deone is driving down Bumpkin Road going $\slimit$ miles per hour when a fawn suddenly appears $\d$ feet away in the middle of the road. \begin{enumerate}
\item {\bf (5 points)} If she slams on her brakes now, how far will her car skid? \vspace{4cm}
\item {\bf (2 points)} Will she avoid hitting the fawn if it freezes in place? Why or why not? Fully explain your reason. \vspace{3cm}
\end{enumerate}


\newpage\def \xis{4}\def \yis{0}\def \nomatcho{[2,5,3,2]}\def \a{2}\def \c{5}\def \b{3}\def \d{-2}\def \polyonesol{8}\def \polytwosol{20}\def \xgoodone{2x^{}}\def \ygoodone{+3y^{}}\def \xgoodtwo{5x^{}}\def \ygoodtwo{-2y^{}}\def \unitize{[1,0,0,0,1,0]}\def \mtem{2}\def \ntem{-4}\def \ptem{-4}\def \qtem{5}\def \m{1}\def \n{-4}\def \p{-4}\def \q{5}\def \polytonesol{4}\def \polyttwosol{-16}\def \xtgoodone{x^{}}\def \ytgoodone{-4y^{}}\def \xtgoodtwo{-4x^{}}\def \ytgoodtwo{+5y^{}}
\item {\bf (6 points)} 
 Solve the system using either substitution or elimination. Write your answer as an ordered pair, if possible. 

$ \begin{cases} \xtgoodone \ytgoodone = \polytonesol \\
\xtgoodtwo \ytgoodtwo = \polyttwosol
\end{cases}$

 \vfill \vfill
\newpage\def \vshift{3}\def \hshift{2}\def \chang{-1}\def \findval{1}\def \yval{5}

 
\item Given the graph of $f(x)$ below, determine the following. {\bf Assume endpoints are included.}
\vspace{2mm}

\begin{tikzpicture}[scale=0.35]
	\def\startx{-10}
	\def\endx{10}
	\def\starty{-10}
	\def\endy{10}
	
	\draw [very thin,step=1,dotted] (\startx-.4, \starty-.4) grid (\endx+.4, \endy+.4);
	\draw[<->, thick] (\startx-.6,0) -- (\endx+.6, 0);
	\draw[<->, thick] (0,\starty-.6) -- (0,\endy+.6);
	\foreach \x in {\startx,...,\endx}
  	\draw[anchor=north] (\x-0.2, 0) node {\tiny $\x$};
	\foreach \y in {\starty,...,-2,-1,1,2,...,\endy}
  	\draw[anchor=east] (0, \y-.2) node {\tiny $\y$};
  	\draw (0.5, \endy+.3) node {$y$};
  	\draw (\endx+.5, 0.3) node {$x$};
	
	\draw (-4+\hshift,\vshift) node[fill,circle,scale=0.35]{} ;
	\draw (4+\hshift,-3+\vshift) node[fill,circle,scale=0.35]{};
	
	\draw[-, samples=100, very thick, domain=-4+\hshift:-2+\hshift]
	plot(\x, {2*(\x-\hshift)+8+\vshift});
	
	\draw[-,samples=100, very thick, domain=-2+\hshift:2+\hshift]
	plot(\x, {(-2)*(\x-\hshift)+\vshift});
	
	\draw[-, samples=100, very thick, domain=2+\hshift:3+\hshift]
	plot(\x, {(-4)+\vshift});
	
	\draw[-, samples=100, very thick, domain=3+\hshift:4+\hshift]
	plot(\x, {(\x-\hshift)-7+\vshift});
\end{tikzpicture} \begin{enumerate}

\item {\bf (2 points)} $f(\findval)$ \vspace{2cm}

\item {\bf (2 points)} The domain. \vspace{2cm}

\item {\bf (2 points)} The range. \vspace{2cm}

\end{enumerate}


\def \radi{3}\def \circumf{9}

 
\item Suppose the following circle has radius $r= \radi$. What is the circumference of the circle?
\vspace{2mm}

\includegraphics[scale = 0.8]{circle}

\vspace{1cm}
\newpage  $ $   \newpage\end{enumerate}\rhead{10006}\lhead{Sample Course}\chead{Sample Assessment - Fall 2023}\graphicspath{{C:/Users/iainc/anaconda3/Randomizer/Sample Course/Sample Assessment/}}\pagenumbering{arabic}\setcounter{page}{1}


\thispagestyle{fancy}

 \noindent Name: Leibniz, Gottfried \vspace{.3cm} \\\noindent Student ID: 8675314 \vspace{.3cm} \\\noindent Instructor: I. Crump \vspace{.3cm} \\\noindent Signature: $\rule{6cm}{0.15mm}$ \vspace{.3cm} \\ 



\vspace{.4cm}

\noindent {\bf Note that the 'assessmentpreface.tex' file in the exams archive folder is read and placed here. This is also where student information is included, either to be replaced with information from the master.csv file or as blanks.}

\vspace{.4cm}

\hrule

\subsection*{Instructions:} \begin{enumerate}[1.]
\item Any cover page materials, per your departmental standards.
\end{enumerate}


\newpage
\begin{enumerate}
\item {\bf (2 points)} 
 Which of these is correct?

\begin{minipage}[t]{1.0\linewidth}\begin{multicols}{4}\begin{itemize}\item[(a)]  Wrong. \item[(e)]  Wrong. \item[(i)]  Correct. \item[(b)]  Wrong. \item[(f)]  Wrong. \item[(j)]  Wrong. \item[(c)]  Wrong. \item[(g)]  Wrong. \item[] \item[(d)]  Wrong. \item[(h)]  Wrong. \item[] \end{itemize}\end{multicols}\end{minipage} \vfill 
\item {\bf (2 points)} 
 Which of these is correct?

\begin{minipage}[t]{1.0\linewidth}\begin{multicols}{4}\begin{itemize}\item[(a)]  Wrong. \item[(e)]  Correct. \item[(i)]  Wrong. \item[(b)]  Wrong. \item[(f)]  Wrong. \item[(j)]  Wrong. \item[(c)]  Wrong. \item[(g)]  Wrong. \item[] \item[(d)]  Wrong. \item[(h)]  Wrong. \item[] \end{itemize}\end{multicols}\end{minipage} \vfill 
\item {\bf (2 points)} 
 Which of these isn't mentally problematic?

\begin{minipage}[t]{1.0\linewidth}\begin{itemize}\item[(a)]  None of the below.  \item[(b)]  I've built a set that contains itself. \item[(c)]  $a \neq a$ \item[(d)]   All of the above. \end{itemize}\end{minipage} \vfill 
\item {\bf (2 points)} 
 The least common denominator for $\displaystyle \frac{x}{x+1}$ and $\displaystyle \frac{1}{x(x-1)}$ is \vspace{.2cm}

\begin{minipage}[t]{1.0\linewidth}\begin{multicols}{2}\begin{itemize}\item[(a)]  $x(x+1)$ \item[(c)]  $(x+1)(x-1)$ \item[(b)]  $x+1$ \item[(d)]  $x(x+1)(x-1)$ \end{itemize}\end{multicols}\end{minipage} \vfill \newpage\def \a{7}\def \atwoone{2}\def \atwotwo{5}\def \atwothree{3}\def \btwothree{7}\def \sumtwothree{10}\def \diftwothree{-4}\def \bigtwothree{300}\def \powtwothree{343}\def \logtwothree{0.5645750340535797}\def \factortwothree{133}\def \atwofour{1.16}\def \btwofour{1.362}\def \tooshorttwofour{10.1}\def \moneytwofour{10.10}\def \longertwofour{10.10000}\def \atwofive{0.12}\def \btwofive{0.12346}\def \athreeone{5}\def \bthreeone{4}\def \setthreetwo{[2, 5, 6]}\def \athreetwo{2}\def \bthreetwo{5}\def \cthreetwo{6}\def \controlthreethree{8}\def \athreethree{3}\def \topthreethree{0}\def \athreefour{3}\def \bthreefour{1}\def \listthreefour{[1, 2, 4, 5]}\def \afourone{8}\def \bfourone{-8}\def \fracfourone{-1}\def \rootfourtwo{12}\def \simplifiedfourtwo{2 \sqrt{3}}\def \sqrtlistfourtwo{[2, 3]}\def \outfourtwo{2}\def \infourtwo{3}\def \wowfourtwo{1}\def \afourthree{0}\def \nicethreefour{3x^{2}-x^{}}\def \nastythreefour{xyz^{3}}\def \cfourthree{4}\def \dfourthree{10}\def \infourthree{4x^{}}\def \outfourthree{+10y^{}}\def \afourfour{1303519}\def \nicefourfour{1,303,519}\def \goodfourfour{1,000,000.12345}\def \badfourfour{1,000,000.1}
\item {\bf (12 points)} 
 General question content.

\vfill 
\def \a{7}\def \atwoone{2}\def \atwotwo{-2}\def \atwothree{2}\def \btwothree{7}\def \sumtwothree{9}\def \diftwothree{-5}\def \bigtwothree{200}\def \powtwothree{49}\def \logtwothree{0.3562071871080222}\def \factortwothree{65}\def \atwofour{1.31}\def \btwofour{1.042}\def \tooshorttwofour{10.1}\def \moneytwofour{10.10}\def \longertwofour{10.10000}\def \atwofive{0.12}\def \btwofive{0.12346}\def \athreeone{6}\def \bthreeone{5}\def \setthreetwo{[2, 5, 6]}\def \athreetwo{2}\def \bthreetwo{5}\def \cthreetwo{6}\def \controlthreethree{-4}\def \athreethree{1}\def \topthreethree{0}\def \athreefour{5}\def \bthreefour{4}\def \listthreefour{[1, 2, 3, 4]}\def \afourone{12}\def \bfourone{-6}\def \fracfourone{-2}\def \rootfourtwo{8}\def \simplifiedfourtwo{2 \sqrt{2}}\def \sqrtlistfourtwo{[2, 2]}\def \outfourtwo{2}\def \infourtwo{2}\def \wowfourtwo{1}\def \afourthree{0}\def \nicethreefour{3x^{2}-x^{}}\def \nastythreefour{xyz^{3}}\def \cfourthree{-4}\def \dfourthree{9}\def \infourthree{-4x^{}}\def \outfourthree{+9y^{}}\def \afourfour{1131024}\def \nicefourfour{1,131,024}\def \goodfourfour{1,000,000.12345}\def \badfourfour{1,000,000.1}
\item {\bf (12 points)} 
 $\setthreetwo$ contains $\athreetwo,\bthreetwo$, and $\cthreetwo$. 
\vfill 
\def \a{7}\def \atwoone{1}\def \atwotwo{-2}\def \atwothree{3}\def \btwothree{8}\def \sumtwothree{11}\def \diftwothree{-5}\def \bigtwothree{300}\def \powtwothree{512}\def \logtwothree{0.5283208335737188}\def \factortwothree{209}\def \atwofour{1.88}\def \btwofour{1.552}\def \tooshorttwofour{10.1}\def \moneytwofour{10.10}\def \longertwofour{10.10000}\def \atwofive{0.12}\def \btwofive{0.12346}\def \athreeone{5}\def \bthreeone{6}\def \setthreetwo{[12, 6, 9]}\def \athreetwo{12}\def \bthreetwo{6}\def \cthreetwo{9}\def \controlthreethree{-4}\def \athreethree{2}\def \topthreethree{1}\def \athreefour{5}\def \bthreefour{1}\def \listthreefour{[1, 2, 3, 4]}\def \afourone{12}\def \bfourone{-6}\def \fracfourone{-2}\def \rootfourtwo{8}\def \simplifiedfourtwo{2 \sqrt{2}}\def \sqrtlistfourtwo{[2, 2]}\def \outfourtwo{2}\def \infourtwo{2}\def \wowfourtwo{1}\def \afourthree{5}\def \nicethreefour{3x^{2}-x^{}+5}\def \nastythreefour{xyz^{3}+5}\def \cfourthree{4}\def \dfourthree{-9}\def \infourthree{4x^{}}\def \outfourthree{-9y^{}}\def \afourfour{1141095}\def \nicefourfour{1,141,095}\def \goodfourfour{1,000,000.12345}\def \badfourfour{1,000,000.1}
\item {\bf (12 points)} 
 \begin{enumerate}
\item $\atwothree + \btwothree = \sumtwothree$
\item $\atwothree - \btwothree = \diftwothree$
\item $\btwothree^{\atwothree} = \powtwothree$
\item $\atwothree \times 100 = \bigtwothree$
\item $\log_{\btwothree}(\atwothree) = \logtwothree$
\item $\factortwothree$ can be factored into two primes.
\end{enumerate} 
\vfill 
\newpage\def \x{88}\def \y{176}\def \L{352}\def \area{15488}
\item {\bf (5 points)} 
 You have $\L$ feet of fencing to enclose a rectangular plot that borders on a river. If you do not fence along the side of the river, what is the largest area that can be enclosed? \\

\begin{tikzpicture}[scale=0.2]
    \draw[snake=coil,segment aspect=0,gray] (0,0)   -- (20,0);
    \draw[black] (4,0) -- (4,-4) -- (16,-4) -- (16,0);
    \draw[gray] (10,1.5) node {River};
    \draw[black] (16,-4) node[anchor=west] {Fence};
\end{tikzpicture}
  
\vfill \vfill \vfill
\def \a{6}\def \b{3}\def \c{-5}\def \r{12}\def \monicpol{x^{}+6}\def \longnbad{3x^{2}+13x^{}-18}\def \anspol{3x^{}-5}
\item {\bf (6 points)} 
 Divide the following using {\bf long division}. Your final answer should be in the form $$ \text{Quotient} + \dfrac{\text{Remainder}}{\text{Divisor}}.$$

\vspace{3mm}

$(\longnbad) \div (\monicpol)$

\vfill  \vfill \vfill
\newpage\def \discount{16}\def \paid{1581.93}\def \rainy{11.89}\def \orcost{1883.25}\def \purcost{1363.73}\def \orrainy{14.15}
\item {\bf (5 points)} 
 A roll-top desk in Elridge Furniture has been marked up \discount\% and is being sold for \$\paid. How much did Elridge Furniture pay the distributer for the desk? (Round to the nearest cent.) Set up an algebraic equation to represent the situation and solve. Show units.

\vfill 
\def \insvar{27}\def \d{60}\def \zerospeed{40.25}\def \slimit{45}\def \s{57}\def \skidd{120.333}\def \safed{75.0}\def \rsafed{75}

 
\item Solve each and include units in your answer. Use the formula: $s = \sqrt{\insvar \cdot d}$ where $s$ is the speed of the car in miles per hour prior to braking, and $d$ is the stopping distance or length of the skid mark, in feet. 

\vspace{3mm}

Deone is driving down Bumpkin Road going $\slimit$ miles per hour when a fawn suddenly appears $\d$ feet away in the middle of the road. \begin{enumerate}
\item {\bf (5 points)} If she slams on her brakes now, how far will her car skid? \vspace{4cm}
\item {\bf (2 points)} Will she avoid hitting the fawn if it freezes in place? Why or why not? Fully explain your reason. \vspace{3cm}
\end{enumerate}


\newpage\def \xis{4}\def \yis{0}\def \nomatcho{[3,5,2,3]}\def \a{3}\def \c{5}\def \b{-2}\def \d{-3}\def \polyonesol{12}\def \polytwosol{20}\def \xgoodone{3x^{}}\def \ygoodone{-2y^{}}\def \xgoodtwo{5x^{}}\def \ygoodtwo{-3y^{}}\def \unitize{[1,0,0,0,1,0]}\def \mtem{-2}\def \ntem{-4}\def \ptem{-4}\def \qtem{3}\def \m{1}\def \n{-4}\def \p{-4}\def \q{3}\def \polytonesol{4}\def \polyttwosol{-16}\def \xtgoodone{x^{}}\def \ytgoodone{-4y^{}}\def \xtgoodtwo{-4x^{}}\def \ytgoodtwo{+3y^{}}
\item {\bf (6 points)} 
 Solve the system using either substitution or elimination. Write your answer as an ordered pair, if possible. 

$ \begin{cases} \xtgoodone \ytgoodone = \polytonesol \\
\xtgoodtwo \ytgoodtwo = \polyttwosol
\end{cases}$

 \vfill \vfill
\newpage\def \vshift{5}\def \hshift{-4}\def \chang{-2}\def \findval{-6}\def \yval{9}

 
\item Given the graph of $f(x)$ below, determine the following. {\bf Assume endpoints are included.}
\vspace{2mm}

\begin{tikzpicture}[scale=0.35]
	\def\startx{-10}
	\def\endx{10}
	\def\starty{-10}
	\def\endy{10}
	
	\draw [very thin,step=1,dotted] (\startx-.4, \starty-.4) grid (\endx+.4, \endy+.4);
	\draw[<->, thick] (\startx-.6,0) -- (\endx+.6, 0);
	\draw[<->, thick] (0,\starty-.6) -- (0,\endy+.6);
	\foreach \x in {\startx,...,\endx}
  	\draw[anchor=north] (\x-0.2, 0) node {\tiny $\x$};
	\foreach \y in {\starty,...,-2,-1,1,2,...,\endy}
  	\draw[anchor=east] (0, \y-.2) node {\tiny $\y$};
  	\draw (0.5, \endy+.3) node {$y$};
  	\draw (\endx+.5, 0.3) node {$x$};
	
	\draw (-4+\hshift,\vshift) node[fill,circle,scale=0.35]{} ;
	\draw (4+\hshift,-3+\vshift) node[fill,circle,scale=0.35]{};
	
	\draw[-, samples=100, very thick, domain=-4+\hshift:-2+\hshift]
	plot(\x, {2*(\x-\hshift)+8+\vshift});
	
	\draw[-,samples=100, very thick, domain=-2+\hshift:2+\hshift]
	plot(\x, {(-2)*(\x-\hshift)+\vshift});
	
	\draw[-, samples=100, very thick, domain=2+\hshift:3+\hshift]
	plot(\x, {(-4)+\vshift});
	
	\draw[-, samples=100, very thick, domain=3+\hshift:4+\hshift]
	plot(\x, {(\x-\hshift)-7+\vshift});
\end{tikzpicture} \begin{enumerate}

\item {\bf (2 points)} $f(\findval)$ \vspace{2cm}

\item {\bf (2 points)} The domain. \vspace{2cm}

\item {\bf (2 points)} The range. \vspace{2cm}

\end{enumerate}


\def \radi{3}\def \circumf{9}

 
\item Suppose the following circle has radius $r= \radi$. What is the circumference of the circle?
\vspace{2mm}

\includegraphics[scale = 0.8]{circle}

\vspace{1cm}
\newpage  $ $   \newpage\end{enumerate}\rhead{10007}\lhead{Sample Course}\chead{Sample Assessment - Fall 2023}\graphicspath{{C:/Users/iainc/anaconda3/Randomizer/Sample Course/Sample Assessment/}}\pagenumbering{arabic}\setcounter{page}{1}


\thispagestyle{fancy}

 \noindent Name: Babbage, Charles \vspace{.3cm} \\\noindent Student ID: 8675315 \vspace{.3cm} \\\noindent Instructor: $\rule{6cm}{0.15mm}$ \vspace{.3cm} \\\noindent Signature: $\rule{6cm}{0.15mm}$ \vspace{.3cm} \\ 



\vspace{.4cm}

\noindent {\bf Note that the 'assessmentpreface.tex' file in the exams archive folder is read and placed here. This is also where student information is included, either to be replaced with information from the master.csv file or as blanks.}

\vspace{.4cm}

\hrule

\subsection*{Instructions:} \begin{enumerate}[1.]
\item Any cover page materials, per your departmental standards.
\end{enumerate}


\newpage
\begin{enumerate}
\item {\bf (2 points)} 
 Which of these is correct?

\begin{minipage}[t]{1.0\linewidth}\begin{multicols}{4}\begin{itemize}\item[(a)]  Wrong. \item[(e)]  Wrong. \item[(i)]  Correct. \item[(b)]  Wrong. \item[(f)]  Wrong. \item[(j)]  Wrong. \item[(c)]  Wrong. \item[(g)]  Wrong. \item[] \item[(d)]  Wrong. \item[(h)]  Wrong. \item[] \end{itemize}\end{multicols}\end{minipage} \vfill 
\item {\bf (2 points)} 
 Which of these is correct?

\begin{minipage}[t]{1.0\linewidth}\begin{multicols}{3}\begin{itemize}\item[(a)]  Wrong. \item[(b)]  Wrong. \item[(c)]  Correct. \end{itemize}\end{multicols}\end{minipage} \vfill 
\item {\bf (2 points)} 
 Which of these is correct?

\begin{minipage}[t]{1.0\linewidth}\begin{multicols}{3}\begin{itemize}\item[(a)]  Wrong. \item[(b)]  Wrong. \item[(c)]  Correct. \end{itemize}\end{multicols}\end{minipage} \vfill 
\item {\bf (2 points)} 
 The least common denominator for $\displaystyle \frac{x}{x+1}$ and $\displaystyle \frac{1}{x(x-1)}$ is \vspace{.2cm}

\begin{minipage}[t]{1.0\linewidth}\begin{multicols}{2}\begin{itemize}\item[(a)]  $x+1$ \item[(c)]  $(x+1)(x-1)$ \item[(b)]  $x(x+1)$ \item[(d)]  $x(x+1)(x-1)$ \end{itemize}\end{multicols}\end{minipage} \vfill \newpage\def \a{7}\def \atwoone{2}\def \atwotwo{3}\def \atwothree{5}\def \btwothree{9}\def \sumtwothree{14}\def \diftwothree{-4}\def \bigtwothree{500}\def \powtwothree{59049}\def \logtwothree{0.7324867603589634}\def \factortwothree{51}\def \atwofour{1.11}\def \btwofour{1.245}\def \tooshorttwofour{10.1}\def \moneytwofour{10.10}\def \longertwofour{10.10000}\def \atwofive{0.12}\def \btwofive{0.12346}\def \athreeone{6}\def \bthreeone{5}\def \setthreetwo{[12, 6, 9]}\def \athreetwo{12}\def \bthreetwo{6}\def \cthreetwo{9}\def \controlthreethree{4}\def \athreethree{1}\def \topthreethree{0}\def \athreefour{5}\def \bthreefour{2}\def \listthreefour{[1, 2, 3, 4]}\def \afourone{12}\def \bfourone{-2}\def \fracfourone{-6}\def \rootfourtwo{12}\def \simplifiedfourtwo{2 \sqrt{3}}\def \sqrtlistfourtwo{[2, 3]}\def \outfourtwo{2}\def \infourtwo{3}\def \wowfourtwo{1}\def \afourthree{5}\def \nicethreefour{3x^{2}-x^{}+5}\def \nastythreefour{xyz^{3}+5}\def \cfourthree{4}\def \dfourthree{10}\def \infourthree{4x^{}}\def \outfourthree{+10y^{}}\def \afourfour{1287992}\def \nicefourfour{1,287,992}\def \goodfourfour{1,000,000.12345}\def \badfourfour{1,000,000.1}
\item {\bf (12 points)} 
  $\nicethreefour = \nastythreefour$ 
\vfill 
\def \a{7}\def \atwoone{1}\def \atwotwo{3}\def \atwothree{2}\def \btwothree{8}\def \sumtwothree{10}\def \diftwothree{-6}\def \bigtwothree{200}\def \powtwothree{64}\def \logtwothree{0.33333333333333337}\def \factortwothree{39}\def \atwofour{1.76}\def \btwofour{1.217}\def \tooshorttwofour{10.1}\def \moneytwofour{10.10}\def \longertwofour{10.10000}\def \atwofive{0.12}\def \btwofive{0.12346}\def \athreeone{6}\def \bthreeone{4}\def \setthreetwo{[3, 7, 7]}\def \athreetwo{3}\def \bthreetwo{7}\def \cthreetwo{7}\def \controlthreethree{4}\def \athreethree{4}\def \topthreethree{0}\def \athreefour{5}\def \bthreefour{4}\def \listthreefour{[1, 2, 3, 4]}\def \afourone{12}\def \bfourone{4}\def \fracfourone{3}\def \rootfourtwo{20}\def \simplifiedfourtwo{2 \sqrt{5}}\def \sqrtlistfourtwo{[2, 5]}\def \outfourtwo{2}\def \infourtwo{5}\def \wowfourtwo{1}\def \afourthree{-5}\def \nicethreefour{3x^{2}-x^{}-5}\def \nastythreefour{xyz^{3}-5}\def \cfourthree{-4}\def \dfourthree{10}\def \infourthree{-4x^{}}\def \outfourthree{+10y^{}}\def \afourfour{1648062}\def \nicefourfour{1,648,062}\def \goodfourfour{1,000,000.12345}\def \badfourfour{1,000,000.1}
\item {\bf (12 points)} 
 $\atwofive < \btwofive$ 
\vfill 
\def \a{7}\def \atwoone{1}\def \atwotwo{3}\def \atwothree{2}\def \btwothree{6}\def \sumtwothree{8}\def \diftwothree{-4}\def \bigtwothree{200}\def \powtwothree{36}\def \logtwothree{0.3868528072345416}\def \factortwothree{65}\def \atwofour{1.84}\def \btwofour{1.28}\def \tooshorttwofour{10.1}\def \moneytwofour{10.10}\def \longertwofour{10.10000}\def \atwofive{0.12}\def \btwofive{0.12346}\def \athreeone{4}\def \bthreeone{3}\def \setthreetwo{[2, 5, 6]}\def \athreetwo{2}\def \bthreetwo{5}\def \cthreetwo{6}\def \controlthreethree{8}\def \athreethree{4}\def \topthreethree{1}\def \athreefour{3}\def \bthreefour{4}\def \listthreefour{[1, 2, 4, 5]}\def \afourone{12}\def \bfourone{4}\def \fracfourone{3}\def \rootfourtwo{20}\def \simplifiedfourtwo{2 \sqrt{5}}\def \sqrtlistfourtwo{[2, 5]}\def \outfourtwo{2}\def \infourtwo{5}\def \wowfourtwo{1}\def \afourthree{0}\def \nicethreefour{3x^{2}-x^{}}\def \nastythreefour{xyz^{3}}\def \cfourthree{-4}\def \dfourthree{10}\def \infourthree{-4x^{}}\def \outfourthree{+10y^{}}\def \afourfour{1107267}\def \nicefourfour{1,107,267}\def \goodfourfour{1,000,000.12345}\def \badfourfour{1,000,000.1}
\item {\bf (12 points)} 
  \goodfourfour $\neq$ \badfourfour 
\vfill 
\newpage\def \x{64}\def \y{128}\def \L{256}\def \area{8192}
\item {\bf (5 points)} 
 You have $\L$ feet of fencing to enclose a rectangular plot that borders on a river. If you do not fence along the side of the river, what is the largest area that can be enclosed? \\

\begin{tikzpicture}[scale=0.2]
    \draw[snake=coil,segment aspect=0,gray] (0,0)   -- (20,0);
    \draw[black] (4,0) -- (4,-4) -- (16,-4) -- (16,0);
    \draw[gray] (10,1.5) node {River};
    \draw[black] (16,-4) node[anchor=west] {Fence};
\end{tikzpicture}
  
\vfill \vfill \vfill
\def \a{6}\def \b{3}\def \c{-7}\def \r{12}\def \monicpol{x^{}+6}\def \longnbad{3x^{2}+11x^{}-30}\def \anspol{3x^{}-7}
\item {\bf (6 points)} 
 Divide the following using {\bf long division}. Your final answer should be in the form $$ \text{Quotient} + \dfrac{\text{Remainder}}{\text{Divisor}}.$$

\vspace{3mm}

$(\longnbad) \div (\monicpol)$

\vfill  \vfill \vfill
\newpage\def \discount{17}\def \paid{1886.56}\def \rainy{10.52}\def \orcost{2272.96}\def \purcost{1612.44}\def \orrainy{12.67}
\item {\bf (5 points)} 
 Elridge Furniture discounts furniture \discount\% to customers paying cash. Jennifer paid \$\paid\ cash for a roll-top desk. What was the original price of the desk? (Round to the nearest cent.) Set up an algebraic equation to represent the situation and solve. Show units.

\vfill 
\def \insvar{27}\def \d{80}\def \zerospeed{46.48}\def \slimit{35}\def \s{56}\def \skidd{116.148}\def \safed{45.37}\def \rsafed{45}

 
\item Solve each and include units in your answer. Use the formula: $s = \sqrt{\insvar \cdot d}$ where $s$ is the speed of the car in miles per hour prior to braking, and $d$ is the stopping distance or length of the skid mark, in feet. 

\vspace{3mm}

Deone is driving down Bumpkin Road going $\slimit$ miles per hour when a fawn suddenly appears $\d$ feet away in the middle of the road. \begin{enumerate}
\item {\bf (5 points)} If she slams on her brakes now, how far will her car skid? \vspace{4cm}
\item {\bf (2 points)} Will she avoid hitting the fawn if it freezes in place? Why or why not? Fully explain your reason. \vspace{3cm}
\end{enumerate}


\newpage\def \xis{5}\def \yis{5}\def \nomatcho{[2,3,5,2]}\def \a{-2}\def \c{-3}\def \b{-5}\def \d{2}\def \polyonesol{-35}\def \polytwosol{-5}\def \xgoodone{-2x^{}}\def \ygoodone{-5y^{}}\def \xgoodtwo{-3x^{}}\def \ygoodtwo{+2y^{}}\def \unitize{[0,0,0,1,0,1]}\def \mtem{-3}\def \ntem{-5}\def \ptem{-5}\def \qtem{3}\def \m{-3}\def \n{-5}\def \p{-5}\def \q{1}\def \polytonesol{-40}\def \polyttwosol{-20}\def \xtgoodone{-3x^{}}\def \ytgoodone{-5y^{}}\def \xtgoodtwo{-5x^{}}\def \ytgoodtwo{+y^{}}
\item {\bf (6 points)} 
 Solve the system using either substitution or elimination. Write your answer as an ordered pair, if possible. 

$ \begin{cases} \xtgoodone \ytgoodone = \polytonesol \\
\xtgoodtwo \ytgoodtwo = \polyttwosol
\end{cases}$

 \vfill \vfill
\newpage\def \vshift{3}\def \hshift{0}\def \chang{-2}\def \findval{-2}\def \yval{7}

 
\item Given the graph of $f(x)$ below, determine the following. {\bf Assume endpoints are included.}
\vspace{2mm}

\begin{tikzpicture}[scale=0.35]
	\def\startx{-10}
	\def\endx{10}
	\def\starty{-10}
	\def\endy{10}
	
	\draw [very thin,step=1,dotted] (\startx-.4, \starty-.4) grid (\endx+.4, \endy+.4);
	\draw[<->, thick] (\startx-.6,0) -- (\endx+.6, 0);
	\draw[<->, thick] (0,\starty-.6) -- (0,\endy+.6);
	\foreach \x in {\startx,...,\endx}
  	\draw[anchor=north] (\x-0.2, 0) node {\tiny $\x$};
	\foreach \y in {\starty,...,-2,-1,1,2,...,\endy}
  	\draw[anchor=east] (0, \y-.2) node {\tiny $\y$};
  	\draw (0.5, \endy+.3) node {$y$};
  	\draw (\endx+.5, 0.3) node {$x$};
	
	\draw (-4+\hshift,\vshift) node[fill,circle,scale=0.35]{} ;
	\draw (4+\hshift,-3+\vshift) node[fill,circle,scale=0.35]{};
	
	\draw[-, samples=100, very thick, domain=-4+\hshift:-2+\hshift]
	plot(\x, {2*(\x-\hshift)+8+\vshift});
	
	\draw[-,samples=100, very thick, domain=-2+\hshift:2+\hshift]
	plot(\x, {(-2)*(\x-\hshift)+\vshift});
	
	\draw[-, samples=100, very thick, domain=2+\hshift:3+\hshift]
	plot(\x, {(-4)+\vshift});
	
	\draw[-, samples=100, very thick, domain=3+\hshift:4+\hshift]
	plot(\x, {(\x-\hshift)-7+\vshift});
\end{tikzpicture} \begin{enumerate}

\item {\bf (2 points)} $f(\findval)$ \vspace{2cm}

\item {\bf (2 points)} The domain. \vspace{2cm}

\item {\bf (2 points)} The range. \vspace{2cm}

\end{enumerate}


\def \radi{5}\def \circumf{25}

 
\item Suppose the following circle has radius $r= \radi$. What is the circumference of the circle?
\vspace{2mm}

\includegraphics[scale = 0.8]{circle}

\vspace{1cm}
\newpage  $ $   \newpage\end{enumerate}\rhead{10008}\lhead{Sample Course}\chead{Sample Assessment - Fall 2023}\graphicspath{{C:/Users/iainc/anaconda3/Randomizer/Sample Course/Sample Assessment/}}\pagenumbering{arabic}\setcounter{page}{1}


\thispagestyle{fancy}

 
\noindent Name: $\rule{6cm}{0.15mm}$

\vspace{.2cm}

\noindent Student ID: $\rule{6cm}{0.15mm}$

\vspace{.2cm}

\noindent Instructor: $\rule{6cm}{0.15mm}$

\vspace{.2cm}

\noindent Signature: $\rule{6cm}{0.15mm}$
 



\vspace{.4cm}

\noindent {\bf Note that the 'assessmentpreface.tex' file in the exams archive folder is read and placed here. This is also where student information is included, either to be replaced with information from the master.csv file or as blanks.}

\vspace{.4cm}

\hrule

\subsection*{Instructions:} \begin{enumerate}[1.]
\item Any cover page materials, per your departmental standards.
\end{enumerate}


\newpage
\begin{enumerate}
\item {\bf (2 points)} 
 Which of these is correct?

\begin{minipage}[t]{1.0\linewidth}\begin{multicols}{4}\begin{itemize}\item[(a)]  Wrong. \item[(e)]  Wrong. \item[(i)]  Correct. \item[(b)]  Wrong. \item[(f)]  Wrong. \item[(j)]  Wrong. \item[(c)]  Wrong. \item[(g)]  Wrong. \item[] \item[(d)]  Wrong. \item[(h)]  Wrong. \item[] \end{itemize}\end{multicols}\end{minipage} \vfill 
\item {\bf (2 points)} 
 Which of these isn't mentally problematic?

\begin{minipage}[t]{1.0\linewidth}\begin{itemize}\item[(a)]  None of the below.  \item[(b)]  I've built a set that contains itself. \item[(c)]  $a \neq a$ \item[(d)]   All of the above. \end{itemize}\end{minipage} \vfill 
\item {\bf (2 points)} 
 Which of these is correct?

\begin{minipage}[t]{1.0\linewidth}\begin{multicols}{4}\begin{itemize}\item[(a)]  Wrong. \item[(e)]  Wrong. \item[(i)]  Wrong. \item[(b)]  Correct. \item[(f)]  Wrong. \item[(j)]  Wrong. \item[(c)]  Wrong. \item[(g)]  Wrong. \item[] \item[(d)]  Wrong. \item[(h)]  Wrong. \item[] \end{itemize}\end{multicols}\end{minipage} \vfill 
\item {\bf (2 points)} 
 The least common denominator for $\displaystyle \frac{x}{x+1}$ and $\displaystyle \frac{1}{x(x-1)}$ is \vspace{.2cm}

\begin{minipage}[t]{1.0\linewidth}\begin{multicols}{2}\begin{itemize}\item[(a)]  $x(x+1)$ \item[(c)]  $x(x+1)(x-1)$ \item[(b)]  $x+1$ \item[(d)]  $(x+1)(x-1)$ \end{itemize}\end{multicols}\end{minipage} \vfill \newpage\def \a{7}\def \atwoone{1}\def \atwotwo{-6}\def \atwothree{4}\def \btwothree{6}\def \sumtwothree{10}\def \diftwothree{-2}\def \bigtwothree{400}\def \powtwothree{1296}\def \logtwothree{0.7737056144690831}\def \factortwothree{26}\def \atwofour{1.02}\def \btwofour{1.175}\def \tooshorttwofour{10.1}\def \moneytwofour{10.10}\def \longertwofour{10.10000}\def \atwofive{0.12}\def \btwofive{0.12346}\def \athreeone{6}\def \bthreeone{7}\def \setthreetwo{[12, 6, 9]}\def \athreetwo{12}\def \bthreetwo{6}\def \cthreetwo{9}\def \controlthreethree{8}\def \athreethree{3}\def \topthreethree{1}\def \athreefour{3}\def \bthreefour{4}\def \listthreefour{[1, 2, 4, 5]}\def \afourone{4}\def \bfourone{-8}\def \fracfourone{\frac{-1}{2}}\def \rootfourtwo{8}\def \simplifiedfourtwo{2 \sqrt{2}}\def \sqrtlistfourtwo{[2, 2]}\def \outfourtwo{2}\def \infourtwo{2}\def \wowfourtwo{1}\def \afourthree{0}\def \nicethreefour{3x^{2}-x^{}}\def \nastythreefour{xyz^{3}}\def \cfourthree{4}\def \dfourthree{-9}\def \infourthree{4x^{}}\def \outfourthree{-9y^{}}\def \afourfour{1145704}\def \nicefourfour{1,145,704}\def \goodfourfour{1,000,000.12345}\def \badfourfour{1,000,000.1}
\item {\bf (12 points)} 
 $\dfrac{1+\sqrt{\rootfourtwo}}{2} = \dfrac{1}{2} + \ifthenelse{\wowfourtwo = 1}{}{\wowfourtwo} \sqrt{\infourtwo}$

\vfill 
\def \a{7}\def \atwoone{3}\def \atwotwo{5}\def \atwothree{3}\def \btwothree{6}\def \sumtwothree{9}\def \diftwothree{-3}\def \bigtwothree{300}\def \powtwothree{216}\def \logtwothree{0.6131471927654585}\def \factortwothree{143}\def \atwofour{1.05}\def \btwofour{1.114}\def \tooshorttwofour{10.1}\def \moneytwofour{10.10}\def \longertwofour{10.10000}\def \atwofive{0.12}\def \btwofive{0.12346}\def \athreeone{6}\def \bthreeone{4}\def \setthreetwo{[12, 6, 9]}\def \athreetwo{12}\def \bthreetwo{6}\def \cthreetwo{9}\def \controlthreethree{-8}\def \athreethree{3}\def \topthreethree{0}\def \athreefour{5}\def \bthreefour{2}\def \listthreefour{[1, 2, 3, 4]}\def \afourone{8}\def \bfourone{-6}\def \fracfourone{\frac{-4}{3}}\def \rootfourtwo{8}\def \simplifiedfourtwo{2 \sqrt{2}}\def \sqrtlistfourtwo{[2, 2]}\def \outfourtwo{2}\def \infourtwo{2}\def \wowfourtwo{1}\def \afourthree{0}\def \nicethreefour{3x^{2}-x^{}}\def \nastythreefour{xyz^{3}}\def \cfourthree{-4}\def \dfourthree{10}\def \infourthree{-4x^{}}\def \outfourthree{+10y^{}}\def \afourfour{1435206}\def \nicefourfour{1,435,206}\def \goodfourfour{1,000,000.12345}\def \badfourfour{1,000,000.1}
\item {\bf (12 points)} 
 I successfully chose then number $\atwotwo$ at random. 
\vfill 
\def \a{7}\def \atwoone{3}\def \atwotwo{5}\def \atwothree{2}\def \btwothree{9}\def \sumtwothree{11}\def \diftwothree{-7}\def \bigtwothree{200}\def \powtwothree{81}\def \logtwothree{0.3154648767857287}\def \factortwothree{85}\def \atwofour{1.58}\def \btwofour{1.673}\def \tooshorttwofour{10.1}\def \moneytwofour{10.10}\def \longertwofour{10.10000}\def \atwofive{0.12}\def \btwofive{0.12346}\def \athreeone{5}\def \bthreeone{4}\def \setthreetwo{[2, 5, 6]}\def \athreetwo{2}\def \bthreetwo{5}\def \cthreetwo{6}\def \controlthreethree{4}\def \athreethree{1}\def \topthreethree{0}\def \athreefour{4}\def \bthreefour{2}\def \listthreefour{[1, 2, 3, 5]}\def \afourone{16}\def \bfourone{4}\def \fracfourone{4}\def \rootfourtwo{12}\def \simplifiedfourtwo{2 \sqrt{3}}\def \sqrtlistfourtwo{[2, 3]}\def \outfourtwo{2}\def \infourtwo{3}\def \wowfourtwo{1}\def \afourthree{5}\def \nicethreefour{3x^{2}-x^{}+5}\def \nastythreefour{xyz^{3}+5}\def \cfourthree{4}\def \dfourthree{10}\def \infourthree{4x^{}}\def \outfourthree{+10y^{}}\def \afourfour{1865628}\def \nicefourfour{1,865,628}\def \goodfourfour{1,000,000.12345}\def \badfourfour{1,000,000.1}
\item {\bf (12 points)} 
  $\infourthree \outfourthree$ 
\vfill 
\newpage\def \x{53}\def \y{106}\def \L{212}\def \area{5618}
\item {\bf (5 points)} 
 You have $\L$ feet of fencing to enclose a rectangular plot that borders on a river. If you do not fence along the side of the river, find the \textbf{dimensions} of the plot that will maximize the area. \\

\begin{tikzpicture}[scale=0.2]
    \draw[snake=coil,segment aspect=0,gray] (0,0)   -- (20,0);
    \draw[black] (4,0) -- (4,-4) -- (16,-4) -- (16,0);
    \draw[gray] (10,1.5) node {River};
    \draw[black] (16,-4) node[anchor=west] {Fence};
\end{tikzpicture}
  
\vfill \vfill \vfill
\def \a{7}\def \b{3}\def \c{-8}\def \r{14}\def \monicpol{x^{}+7}\def \longnbad{3x^{2}+13x^{}-42}\def \anspol{3x^{}-8}
\item {\bf (6 points)} 
 Divide the following using {\bf long division}. Your final answer should be in the form $$ \text{Quotient} + \dfrac{\text{Remainder}}{\text{Divisor}}.$$

\vspace{3mm}

$(\longnbad) \div (\monicpol)$

\vfill  \vfill \vfill
\newpage\def \discount{13}\def \paid{1893.58}\def \rainy{9.75}\def \orcost{2176.53}\def \purcost{1675.73}\def \orrainy{11.21}
\item {\bf (5 points)} 
 In April of this year, Greenfield received \rainy\ inches of rain. This was \discount\% less than the amount recorded in April of 2010. How much rain did Greenfield  receive in April 2010? Set up an algebraic equation to represent the situation and solve. Show units.

\vfill 
\def \insvar{24}\def \d{75}\def \zerospeed{42.43}\def \slimit{30}\def \s{55}\def \skidd{126.042}\def \safed{37.5}\def \rsafed{37}

 
\item Solve each and include units in your answer. Use the formula: $s = \sqrt{\insvar \cdot d}$ where $s$ is the speed of the car in miles per hour prior to braking, and $d$ is the stopping distance or length of the skid mark, in feet. 

\vspace{3mm}

Deone is driving down Bumpkin Road going $\slimit$ miles per hour when a fawn suddenly appears $\d$ feet away in the middle of the road. \begin{enumerate}
\item {\bf (5 points)} If she slams on her brakes now, how far will her car skid? \vspace{4cm}
\item {\bf (2 points)} Will she avoid hitting the fawn if it freezes in place? Why or why not? Fully explain your reason. \vspace{3cm}
\end{enumerate}


\newpage\def \xis{-4}\def \yis{-5}\def \nomatcho{[5,3,2,5]}\def \a{5}\def \c{-3}\def \b{2}\def \d{5}\def \polyonesol{-30}\def \polytwosol{-13}\def \xgoodone{5x^{}}\def \ygoodone{+2y^{}}\def \xgoodtwo{-3x^{}}\def \ygoodtwo{+5y^{}}\def \unitize{[0,0,1,0,0,1]}\def \mtem{3}\def \ntem{5}\def \ptem{-4}\def \qtem{-3}\def \m{3}\def \n{5}\def \p{1}\def \q{-3}\def \polytonesol{-37}\def \polyttwosol{11}\def \xtgoodone{3x^{}}\def \ytgoodone{+5y^{}}\def \xtgoodtwo{x^{}}\def \ytgoodtwo{-3y^{}}
\item {\bf (6 points)} 
 Solve the system using either substitution or elimination. Write your answer as an ordered pair, if possible. 

$ \begin{cases} \xtgoodone \ytgoodone = \polytonesol \\
\xtgoodtwo \ytgoodtwo = \polyttwosol
\end{cases}$

 \vfill \vfill
\newpage\def \vshift{-5}\def \hshift{-4}\def \chang{1}\def \findval{-3}\def \yval{-7}

 
\item Given the graph of $f(x)$ below, determine the following. {\bf Assume endpoints are included.}
\vspace{2mm}

\begin{tikzpicture}[scale=0.35]
	\def\startx{-10}
	\def\endx{10}
	\def\starty{-10}
	\def\endy{10}
	
	\draw [very thin,step=1,dotted] (\startx-.4, \starty-.4) grid (\endx+.4, \endy+.4);
	\draw[<->, thick] (\startx-.6,0) -- (\endx+.6, 0);
	\draw[<->, thick] (0,\starty-.6) -- (0,\endy+.6);
	\foreach \x in {\startx,...,\endx}
  	\draw[anchor=north] (\x-0.2, 0) node {\tiny $\x$};
	\foreach \y in {\starty,...,-2,-1,1,2,...,\endy}
  	\draw[anchor=east] (0, \y-.2) node {\tiny $\y$};
  	\draw (0.5, \endy+.3) node {$y$};
  	\draw (\endx+.5, 0.3) node {$x$};
	
	\draw (-4+\hshift,\vshift) node[fill,circle,scale=0.35]{} ;
	\draw (4+\hshift,-3+\vshift) node[fill,circle,scale=0.35]{};
	
	\draw[-, samples=100, very thick, domain=-4+\hshift:-2+\hshift]
	plot(\x, {2*(\x-\hshift)+8+\vshift});
	
	\draw[-,samples=100, very thick, domain=-2+\hshift:2+\hshift]
	plot(\x, {(-2)*(\x-\hshift)+\vshift});
	
	\draw[-, samples=100, very thick, domain=2+\hshift:3+\hshift]
	plot(\x, {(-4)+\vshift});
	
	\draw[-, samples=100, very thick, domain=3+\hshift:4+\hshift]
	plot(\x, {(\x-\hshift)-7+\vshift});
\end{tikzpicture} \begin{enumerate}

\item {\bf (2 points)} $f(\findval)$ \vspace{2cm}

\item {\bf (2 points)} The domain. \vspace{2cm}

\item {\bf (2 points)} The range. \vspace{2cm}

\end{enumerate}


\def \radi{4}\def \circumf{16}

 
\item Suppose the following circle has radius $r= \radi$. What is the circumference of the circle?
\vspace{2mm}

\includegraphics[scale = 0.8]{circle}

\vspace{1cm}
\newpage  $ $   \newpage\end{enumerate}\rhead{10009}\lhead{Sample Course}\chead{Sample Assessment - Fall 2023}\graphicspath{{C:/Users/iainc/anaconda3/Randomizer/Sample Course/Sample Assessment/}}\pagenumbering{arabic}\setcounter{page}{1}


\thispagestyle{fancy}

 
\noindent Name: $\rule{6cm}{0.15mm}$

\vspace{.2cm}

\noindent Student ID: $\rule{6cm}{0.15mm}$

\vspace{.2cm}

\noindent Instructor: $\rule{6cm}{0.15mm}$

\vspace{.2cm}

\noindent Signature: $\rule{6cm}{0.15mm}$
 



\vspace{.4cm}

\noindent {\bf Note that the 'assessmentpreface.tex' file in the exams archive folder is read and placed here. This is also where student information is included, either to be replaced with information from the master.csv file or as blanks.}

\vspace{.4cm}

\hrule

\subsection*{Instructions:} \begin{enumerate}[1.]
\item Any cover page materials, per your departmental standards.
\end{enumerate}


\newpage
\begin{enumerate}
\item {\bf (2 points)} 
 Which of these is correct?

\begin{minipage}[t]{1.0\linewidth}\begin{multicols}{4}\begin{itemize}\item[(a)]  Wrong. \item[(e)]  Wrong. \item[(i)]  Wrong. \item[(b)]  Wrong. \item[(f)]  Wrong. \item[(j)]  Correct. \item[(c)]  Wrong. \item[(g)]  Wrong. \item[] \item[(d)]  Wrong. \item[(h)]  Wrong. \item[] \end{itemize}\end{multicols}\end{minipage} \vfill 
\item {\bf (2 points)} 
 Which of these isn't mentally problematic?

\begin{minipage}[t]{1.0\linewidth}\begin{itemize}\item[(a)]  None of the below.  \item[(b)]  I've built a set that contains itself. \item[(c)]  $a \neq a$ \item[(d)]   All of the above. \end{itemize}\end{minipage} \vfill 
\item {\bf (2 points)} 
 Which of these is correct?

\begin{minipage}[t]{1.0\linewidth}\begin{multicols}{3}\begin{itemize}\item[(a)]  Wrong. \item[(b)]  Correct. \item[(c)]  Wrong. \end{itemize}\end{multicols}\end{minipage} \vfill 
\item {\bf (2 points)} 
 The least common denominator for $\displaystyle \frac{x}{x+1}$ and $\displaystyle \frac{1}{x-1}$ is \vspace{.2cm}

\begin{minipage}[t]{1.0\linewidth}\begin{multicols}{2}\begin{itemize}\item[(a)]  $x(x+1)$ \item[(c)]  $(x+1)(x-1)$ \item[(b)]  $x+1$ \item[(d)]  $x(x+1)(x-1)$ \end{itemize}\end{multicols}\end{minipage} \vfill \newpage\def \a{7}\def \atwoone{1}\def \atwotwo{3}\def \atwothree{5}\def \btwothree{7}\def \sumtwothree{12}\def \diftwothree{-2}\def \bigtwothree{500}\def \powtwothree{16807}\def \logtwothree{0.8270874753469162}\def \factortwothree{51}\def \atwofour{1.24}\def \btwofour{1.451}\def \tooshorttwofour{10.1}\def \moneytwofour{10.10}\def \longertwofour{10.10000}\def \atwofive{0.12}\def \btwofive{0.12346}\def \athreeone{4}\def \bthreeone{6}\def \setthreetwo{[12, 6, 9]}\def \athreetwo{12}\def \bthreetwo{6}\def \cthreetwo{9}\def \controlthreethree{-5}\def \athreethree{3}\def \topthreethree{0}\def \athreefour{5}\def \bthreefour{2}\def \listthreefour{[1, 2, 3, 4]}\def \afourone{4}\def \bfourone{-8}\def \fracfourone{\frac{-1}{2}}\def \rootfourtwo{8}\def \simplifiedfourtwo{2 \sqrt{2}}\def \sqrtlistfourtwo{[2, 2]}\def \outfourtwo{2}\def \infourtwo{2}\def \wowfourtwo{1}\def \afourthree{0}\def \nicethreefour{3x^{2}-x^{}}\def \nastythreefour{xyz^{3}}\def \cfourthree{-4}\def \dfourthree{9}\def \infourthree{-4x^{}}\def \outfourthree{+9y^{}}\def \afourfour{1159420}\def \nicefourfour{1,159,420}\def \goodfourfour{1,000,000.12345}\def \badfourfour{1,000,000.1}
\item {\bf (12 points)} 
 Any preamble. \begin{enumerate}
\item A first part. \vspace{2cm}
\item A second part.
\end{enumerate}

\vfill 
\def \a{7}\def \atwoone{1}\def \atwotwo{3}\def \atwothree{2}\def \btwothree{6}\def \sumtwothree{8}\def \diftwothree{-4}\def \bigtwothree{200}\def \powtwothree{36}\def \logtwothree{0.3868528072345416}\def \factortwothree{119}\def \atwofour{1.99}\def \btwofour{1.722}\def \tooshorttwofour{10.1}\def \moneytwofour{10.10}\def \longertwofour{10.10000}\def \atwofive{0.12}\def \btwofive{0.12346}\def \athreeone{6}\def \bthreeone{8}\def \setthreetwo{[12, 6, 9]}\def \athreetwo{12}\def \bthreetwo{6}\def \cthreetwo{9}\def \controlthreethree{5}\def \athreethree{3}\def \topthreethree{0}\def \athreefour{5}\def \bthreefour{2}\def \listthreefour{[1, 2, 3, 4]}\def \afourone{16}\def \bfourone{-2}\def \fracfourone{-8}\def \rootfourtwo{12}\def \simplifiedfourtwo{2 \sqrt{3}}\def \sqrtlistfourtwo{[2, 3]}\def \outfourtwo{2}\def \infourtwo{3}\def \wowfourtwo{1}\def \afourthree{5}\def \nicethreefour{3x^{2}-x^{}+5}\def \nastythreefour{xyz^{3}+5}\def \cfourthree{4}\def \dfourthree{9}\def \infourthree{4x^{}}\def \outfourthree{+9y^{}}\def \afourfour{1203896}\def \nicefourfour{1,203,896}\def \goodfourfour{1,000,000.12345}\def \badfourfour{1,000,000.1}
\item {\bf (12 points)} 
 $\dfrac{\afourone}{\bfourone} = \fracfourone$ or $\displaystyle \fracfourone$ 

\vfill 
\def \a{7}\def \atwoone{2}\def \atwotwo{3}\def \atwothree{1}\def \btwothree{7}\def \sumtwothree{8}\def \diftwothree{-6}\def \bigtwothree{100}\def \powtwothree{7}\def \logtwothree{0.0}\def \factortwothree{38}\def \atwofour{1.57}\def \btwofour{1.496}\def \tooshorttwofour{10.1}\def \moneytwofour{10.10}\def \longertwofour{10.10000}\def \atwofive{0.12}\def \btwofive{0.12346}\def \athreeone{4}\def \bthreeone{3}\def \setthreetwo{[2, 5, 6]}\def \athreetwo{2}\def \bthreetwo{5}\def \cthreetwo{6}\def \controlthreethree{-8}\def \athreethree{1}\def \topthreethree{1}\def \athreefour{4}\def \bthreefour{3}\def \listthreefour{[1, 2, 3, 5]}\def \afourone{4}\def \bfourone{-2}\def \fracfourone{-2}\def \rootfourtwo{12}\def \simplifiedfourtwo{2 \sqrt{3}}\def \sqrtlistfourtwo{[2, 3]}\def \outfourtwo{2}\def \infourtwo{3}\def \wowfourtwo{1}\def \afourthree{-5}\def \nicethreefour{3x^{2}-x^{}-5}\def \nastythreefour{xyz^{3}-5}\def \cfourthree{4}\def \dfourthree{10}\def \infourthree{4x^{}}\def \outfourthree{+10y^{}}\def \afourfour{1482487}\def \nicefourfour{1,482,487}\def \goodfourfour{1,000,000.12345}\def \badfourfour{1,000,000.1}
\item {\bf (12 points)} 
 $\setthreetwo$ contains $\athreetwo,\bthreetwo$, and $\cthreetwo$. 
\vfill 
\newpage\def \x{52}\def \y{104}\def \L{208}\def \area{5408}
\item {\bf (5 points)} 
 You have $\L$ feet of fencing to enclose a rectangular plot that borders on a river. If you do not fence along the side of the river, find the \textbf{dimensions} of the plot that will maximize the area. \\

\begin{tikzpicture}[scale=0.2]
    \draw[snake=coil,segment aspect=0,gray] (0,0)   -- (20,0);
    \draw[black] (4,0) -- (4,-4) -- (16,-4) -- (16,0);
    \draw[gray] (10,1.5) node {River};
    \draw[black] (16,-4) node[anchor=west] {Fence};
\end{tikzpicture}
  
\vfill \vfill \vfill
\def \a{4}\def \b{3}\def \c{-5}\def \r{12}\def \monicpol{x^{}+4}\def \longnbad{3x^{2}+7x^{}-8}\def \anspol{3x^{}-5}
\item {\bf (6 points)} 
 Divide the following using {\bf long division}. Your final answer should be in the form $$ \text{Quotient} + \dfrac{\text{Remainder}}{\text{Divisor}}.$$

\vspace{3mm}

$(\longnbad) \div (\monicpol)$

\vfill  \vfill \vfill
\newpage\def \discount{14}\def \paid{1302.78}\def \rainy{11.75}\def \orcost{1514.86}\def \purcost{1142.79}\def \orrainy{13.66}
\item {\bf (5 points)} 
 A roll-top desk in Elridge Furniture has been marked up \discount\% and is being sold for \$\paid. How much did Elridge Furniture pay the distributer for the desk? (Round to the nearest cent.) Set up an algebraic equation to represent the situation and solve. Show units.

\vfill 
\def \insvar{21}\def \d{60}\def \zerospeed{35.5}\def \slimit{25}\def \s{46}\def \skidd{100.762}\def \safed{29.762}\def \rsafed{29}

 
\item Solve each and include units in your answer. Use the formula: $s = \sqrt{\insvar \cdot d}$ where $s$ is the speed of the car in miles per hour prior to braking, and $d$ is the stopping distance or length of the skid mark, in feet. 

\vspace{3mm}

Deone is driving down Bumpkin Road going $\slimit$ miles per hour when a fawn suddenly appears $\d$ feet away in the middle of the road. \begin{enumerate}
\item {\bf (5 points)} If she slams on her brakes now, how far will her car skid? \vspace{4cm}
\item {\bf (2 points)} Will she avoid hitting the fawn if it freezes in place? Why or why not? Fully explain your reason. \vspace{3cm}
\end{enumerate}


\newpage\def \xis{-1}\def \yis{0}\def \nomatcho{[3,5,2,3]}\def \a{-3}\def \c{5}\def \b{2}\def \d{3}\def \polyonesol{3}\def \polytwosol{-5}\def \xgoodone{-3x^{}}\def \ygoodone{+2y^{}}\def \xgoodtwo{5x^{}}\def \ygoodtwo{+3y^{}}\def \unitize{[1,0,0,0,1,0]}\def \mtem{2}\def \ntem{5}\def \ptem{-4}\def \qtem{-3}\def \m{1}\def \n{5}\def \p{-4}\def \q{-3}\def \polytonesol{-1}\def \polyttwosol{4}\def \xtgoodone{x^{}}\def \ytgoodone{+5y^{}}\def \xtgoodtwo{-4x^{}}\def \ytgoodtwo{-3y^{}}
\item {\bf (6 points)} 
 Solve the system using either substitution or elimination. Write your answer as an ordered pair, if possible. 

$ \begin{cases} \xtgoodone \ytgoodone = \polytonesol \\
\xtgoodtwo \ytgoodtwo = \polyttwosol
\end{cases}$

 \vfill \vfill
\newpage\def \vshift{3}\def \hshift{2}\def \chang{-1}\def \findval{1}\def \yval{5}

 
\item Given the graph of $f(x)$ below, determine the following. {\bf Assume endpoints are included.}
\vspace{2mm}

\begin{tikzpicture}[scale=0.35]
	\def\startx{-10}
	\def\endx{10}
	\def\starty{-10}
	\def\endy{10}
	
	\draw [very thin,step=1,dotted] (\startx-.4, \starty-.4) grid (\endx+.4, \endy+.4);
	\draw[<->, thick] (\startx-.6,0) -- (\endx+.6, 0);
	\draw[<->, thick] (0,\starty-.6) -- (0,\endy+.6);
	\foreach \x in {\startx,...,\endx}
  	\draw[anchor=north] (\x-0.2, 0) node {\tiny $\x$};
	\foreach \y in {\starty,...,-2,-1,1,2,...,\endy}
  	\draw[anchor=east] (0, \y-.2) node {\tiny $\y$};
  	\draw (0.5, \endy+.3) node {$y$};
  	\draw (\endx+.5, 0.3) node {$x$};
	
	\draw (-4+\hshift,\vshift) node[fill,circle,scale=0.35]{} ;
	\draw (4+\hshift,-3+\vshift) node[fill,circle,scale=0.35]{};
	
	\draw[-, samples=100, very thick, domain=-4+\hshift:-2+\hshift]
	plot(\x, {2*(\x-\hshift)+8+\vshift});
	
	\draw[-,samples=100, very thick, domain=-2+\hshift:2+\hshift]
	plot(\x, {(-2)*(\x-\hshift)+\vshift});
	
	\draw[-, samples=100, very thick, domain=2+\hshift:3+\hshift]
	plot(\x, {(-4)+\vshift});
	
	\draw[-, samples=100, very thick, domain=3+\hshift:4+\hshift]
	plot(\x, {(\x-\hshift)-7+\vshift});
\end{tikzpicture} \begin{enumerate}

\item {\bf (2 points)} $f(\findval)$ \vspace{2cm}

\item {\bf (2 points)} The domain. \vspace{2cm}

\item {\bf (2 points)} The range. \vspace{2cm}

\end{enumerate}


\def \radi{5}\def \circumf{25}

 
\item Suppose the following circle has radius $r= \radi$. What is the circumference of the circle?
\vspace{2mm}

\includegraphics[scale = 0.8]{circle}

\vspace{1cm}
\newpage  $ $   \newpage\end{enumerate}\rhead{10010}\lhead{Sample Course}\chead{Sample Assessment - Fall 2023}\graphicspath{{C:/Users/iainc/anaconda3/Randomizer/Sample Course/Sample Assessment/}}\pagenumbering{arabic}\setcounter{page}{1}


\thispagestyle{fancy}

 
\noindent Name: $\rule{6cm}{0.15mm}$

\vspace{.2cm}

\noindent Student ID: $\rule{6cm}{0.15mm}$

\vspace{.2cm}

\noindent Instructor: $\rule{6cm}{0.15mm}$

\vspace{.2cm}

\noindent Signature: $\rule{6cm}{0.15mm}$
 



\vspace{.4cm}

\noindent {\bf Note that the 'assessmentpreface.tex' file in the exams archive folder is read and placed here. This is also where student information is included, either to be replaced with information from the master.csv file or as blanks.}

\vspace{.4cm}

\hrule

\subsection*{Instructions:} \begin{enumerate}[1.]
\item Any cover page materials, per your departmental standards.
\end{enumerate}


\newpage
\begin{enumerate}
\item {\bf (2 points)} 
 Which of these isn't mentally problematic?

\begin{minipage}[t]{1.0\linewidth}\begin{itemize}\item[(a)]  None of the below.  \item[(b)]  I've built a set that contains itself. \item[(c)]  $a \neq a$ \item[(d)]   All of the above. \end{itemize}\end{minipage} \vfill 
\item {\bf (2 points)} 
 Which of these is correct?

\begin{minipage}[t]{1.0\linewidth}\begin{multicols}{4}\begin{itemize}\item[(a)]  Correct. \item[(e)]  Wrong. \item[(i)]  Wrong. \item[(b)]  Wrong. \item[(f)]  Wrong. \item[(j)]  Wrong. \item[(c)]  Wrong. \item[(g)]  Wrong. \item[] \item[(d)]  Wrong. \item[(h)]  Wrong. \item[] \end{itemize}\end{multicols}\end{minipage} \vfill 
\item {\bf (2 points)} 
 Which of these is correct?

\begin{minipage}[t]{1.0\linewidth}\begin{multicols}{3}\begin{itemize}\item[(a)]  Wrong. \item[(b)]  Wrong. \item[(c)]  Correct. \end{itemize}\end{multicols}\end{minipage} \vfill 
\item {\bf (2 points)} 
 The least common denominator for $\displaystyle \frac{x}{x+1}$ and $\displaystyle \frac{1}{x-1}$ is \vspace{.2cm}

\begin{minipage}[t]{1.0\linewidth}\begin{multicols}{2}\begin{itemize}\item[(a)]  $(x+1)(x-1)$ \item[(c)]  $x(x+1)$ \item[(b)]  $x+1$ \item[(d)]  $x(x+1)(x-1)$ \end{itemize}\end{multicols}\end{minipage} \vfill \newpage\def \a{7}\def \atwoone{3}\def \atwotwo{-2}\def \atwothree{2}\def \btwothree{9}\def \sumtwothree{11}\def \diftwothree{-7}\def \bigtwothree{200}\def \powtwothree{81}\def \logtwothree{0.3154648767857287}\def \factortwothree{34}\def \atwofour{1.89}\def \btwofour{1.232}\def \tooshorttwofour{10.1}\def \moneytwofour{10.10}\def \longertwofour{10.10000}\def \atwofive{0.12}\def \btwofive{0.12346}\def \athreeone{5}\def \bthreeone{6}\def \setthreetwo{[3, 7, 7]}\def \athreetwo{3}\def \bthreetwo{7}\def \cthreetwo{7}\def \controlthreethree{-8}\def \athreethree{2}\def \topthreethree{1}\def \athreefour{5}\def \bthreefour{2}\def \listthreefour{[1, 2, 3, 4]}\def \afourone{4}\def \bfourone{4}\def \fracfourone{1}\def \rootfourtwo{12}\def \simplifiedfourtwo{2 \sqrt{3}}\def \sqrtlistfourtwo{[2, 3]}\def \outfourtwo{2}\def \infourtwo{3}\def \wowfourtwo{1}\def \afourthree{5}\def \nicethreefour{3x^{2}-x^{}+5}\def \nastythreefour{xyz^{3}+5}\def \cfourthree{-4}\def \dfourthree{-10}\def \infourthree{-4x^{}}\def \outfourthree{-10y^{}}\def \afourfour{1923720}\def \nicefourfour{1,923,720}\def \goodfourfour{1,000,000.12345}\def \badfourfour{1,000,000.1}
\item {\bf (12 points)} 
 \begin{enumerate}
\item If I add $\controlthreethree$ to $y=x^2$, the graph shifts \ifthenelse{\controlthreethree >0}{up}{down}.
\item $y = \ifthenelse{\athreethree = 1}{4^x}{\athreethree(4^x)}$
\item $y = \ifthenelse{\topthreethree = 1}{x}{\frac{1}{x}}$
\end{enumerate}

\vfill 
\def \a{7}\def \atwoone{1}\def \atwotwo{5}\def \atwothree{3}\def \btwothree{7}\def \sumtwothree{10}\def \diftwothree{-4}\def \bigtwothree{300}\def \powtwothree{343}\def \logtwothree{0.5645750340535797}\def \factortwothree{51}\def \atwofour{1.38}\def \btwofour{1.843}\def \tooshorttwofour{10.1}\def \moneytwofour{10.10}\def \longertwofour{10.10000}\def \atwofive{0.12}\def \btwofive{0.12346}\def \athreeone{6}\def \bthreeone{8}\def \setthreetwo{[2, 5, 6]}\def \athreetwo{2}\def \bthreetwo{5}\def \cthreetwo{6}\def \controlthreethree{-4}\def \athreethree{4}\def \topthreethree{0}\def \athreefour{4}\def \bthreefour{1}\def \listthreefour{[1, 2, 3, 5]}\def \afourone{12}\def \bfourone{4}\def \fracfourone{3}\def \rootfourtwo{20}\def \simplifiedfourtwo{2 \sqrt{5}}\def \sqrtlistfourtwo{[2, 5]}\def \outfourtwo{2}\def \infourtwo{5}\def \wowfourtwo{1}\def \afourthree{5}\def \nicethreefour{3x^{2}-x^{}+5}\def \nastythreefour{xyz^{3}+5}\def \cfourthree{4}\def \dfourthree{9}\def \infourthree{4x^{}}\def \outfourthree{+9y^{}}\def \afourfour{1855192}\def \nicefourfour{1,855,192}\def \goodfourfour{1,000,000.12345}\def \badfourfour{1,000,000.1}
\item {\bf (12 points)} 
  \goodfourfour $\neq$ \badfourfour 
\vfill 
\def \a{7}\def \atwoone{1}\def \atwotwo{-2}\def \atwothree{1}\def \btwothree{9}\def \sumtwothree{10}\def \diftwothree{-8}\def \bigtwothree{100}\def \powtwothree{9}\def \logtwothree{0.0}\def \factortwothree{187}\def \atwofour{1.32}\def \btwofour{1.265}\def \tooshorttwofour{10.1}\def \moneytwofour{10.10}\def \longertwofour{10.10000}\def \atwofive{0.12}\def \btwofive{0.12346}\def \athreeone{6}\def \bthreeone{5}\def \setthreetwo{[12, 6, 9]}\def \athreetwo{12}\def \bthreetwo{6}\def \cthreetwo{9}\def \controlthreethree{4}\def \athreethree{3}\def \topthreethree{0}\def \athreefour{3}\def \bthreefour{2}\def \listthreefour{[1, 2, 4, 5]}\def \afourone{4}\def \bfourone{-8}\def \fracfourone{\frac{-1}{2}}\def \rootfourtwo{8}\def \simplifiedfourtwo{2 \sqrt{2}}\def \sqrtlistfourtwo{[2, 2]}\def \outfourtwo{2}\def \infourtwo{2}\def \wowfourtwo{1}\def \afourthree{-5}\def \nicethreefour{3x^{2}-x^{}-5}\def \nastythreefour{xyz^{3}-5}\def \cfourthree{-4}\def \dfourthree{-10}\def \infourthree{-4x^{}}\def \outfourthree{-10y^{}}\def \afourfour{1024902}\def \nicefourfour{1,024,902}\def \goodfourfour{1,000,000.12345}\def \badfourfour{1,000,000.1}
\item {\bf (12 points)} 
  $\nicethreefour = \nastythreefour$ 
\vfill 
\newpage\def \x{86}\def \y{172}\def \L{344}\def \area{14792}
\item {\bf (5 points)} 
 You have $\L$ feet of fencing to enclose a rectangular plot that borders on a river. If you do not fence along the side of the river, what is the largest area that can be enclosed? \\

\begin{tikzpicture}[scale=0.2]
    \draw[snake=coil,segment aspect=0,gray] (0,0)   -- (20,0);
    \draw[black] (4,0) -- (4,-4) -- (16,-4) -- (16,0);
    \draw[gray] (10,1.5) node {River};
    \draw[black] (16,-4) node[anchor=west] {Fence};
\end{tikzpicture}
  
\vfill \vfill \vfill
\def \a{5}\def \b{3}\def \c{-8}\def \r{11}\def \monicpol{x^{}+5}\def \longnbad{3x^{2}+7x^{}-29}\def \anspol{3x^{}-8}
\item {\bf (6 points)} 
 Divide the following using {\bf long division}. Your final answer should be in the form $$ \text{Quotient} + \dfrac{\text{Remainder}}{\text{Divisor}}.$$

\vspace{3mm}

$(\longnbad) \div (\monicpol)$

\vfill  \vfill \vfill
\newpage\def \discount{15}\def \paid{1474.49}\def \rainy{12.05}\def \orcost{1734.69}\def \purcost{1282.17}\def \orrainy{14.18}
\item {\bf (5 points)} 
 A roll-top desk in Elridge Furniture has been marked up \discount\% and is being sold for \$\paid. How much did Elridge Furniture pay the distributer for the desk? (Round to the nearest cent.) Set up an algebraic equation to represent the situation and solve. Show units.

\vfill 
\def \insvar{24}\def \d{70}\def \zerospeed{40.99}\def \slimit{35}\def \s{45}\def \skidd{84.375}\def \safed{51.042}\def \rsafed{51}

 
\item Solve each and include units in your answer. Use the formula: $s = \sqrt{\insvar \cdot d}$ where $s$ is the speed of the car in miles per hour prior to braking, and $d$ is the stopping distance or length of the skid mark, in feet. 

\vspace{3mm}

Deone is driving down Bumpkin Road going $\slimit$ miles per hour when a fawn suddenly appears $\d$ feet away in the middle of the road. \begin{enumerate}
\item {\bf (5 points)} If she slams on her brakes now, how far will her car skid? \vspace{4cm}
\item {\bf (2 points)} Will she avoid hitting the fawn if it freezes in place? Why or why not? Fully explain your reason. \vspace{3cm}
\end{enumerate}


\newpage\def \xis{3}\def \yis{2}\def \nomatcho{[2,5,3,2]}\def \a{-2}\def \c{5}\def \b{-3}\def \d{2}\def \polyonesol{-12}\def \polytwosol{19}\def \xgoodone{-2x^{}}\def \ygoodone{-3y^{}}\def \xgoodtwo{5x^{}}\def \ygoodtwo{+2y^{}}\def \unitize{[0,0,0,1,0,1]}\def \mtem{-2}\def \ntem{4}\def \ptem{-4}\def \qtem{5}\def \m{-2}\def \n{4}\def \p{-4}\def \q{1}\def \polytonesol{2}\def \polyttwosol{-10}\def \xtgoodone{-2x^{}}\def \ytgoodone{+4y^{}}\def \xtgoodtwo{-4x^{}}\def \ytgoodtwo{+y^{}}
\item {\bf (6 points)} 
 Solve the system using either substitution or elimination. Write your answer as an ordered pair, if possible. 

$ \begin{cases} \xtgoodone \ytgoodone = \polytonesol \\
\xtgoodtwo \ytgoodtwo = \polyttwosol
\end{cases}$

 \vfill \vfill
\newpage\def \vshift{-3}\def \hshift{2}\def \chang{1}\def \findval{3}\def \yval{-5}

 
\item Given the graph of $f(x)$ below, determine the following. {\bf Assume endpoints are included.}
\vspace{2mm}

\begin{tikzpicture}[scale=0.35]
	\def\startx{-10}
	\def\endx{10}
	\def\starty{-10}
	\def\endy{10}
	
	\draw [very thin,step=1,dotted] (\startx-.4, \starty-.4) grid (\endx+.4, \endy+.4);
	\draw[<->, thick] (\startx-.6,0) -- (\endx+.6, 0);
	\draw[<->, thick] (0,\starty-.6) -- (0,\endy+.6);
	\foreach \x in {\startx,...,\endx}
  	\draw[anchor=north] (\x-0.2, 0) node {\tiny $\x$};
	\foreach \y in {\starty,...,-2,-1,1,2,...,\endy}
  	\draw[anchor=east] (0, \y-.2) node {\tiny $\y$};
  	\draw (0.5, \endy+.3) node {$y$};
  	\draw (\endx+.5, 0.3) node {$x$};
	
	\draw (-4+\hshift,\vshift) node[fill,circle,scale=0.35]{} ;
	\draw (4+\hshift,-3+\vshift) node[fill,circle,scale=0.35]{};
	
	\draw[-, samples=100, very thick, domain=-4+\hshift:-2+\hshift]
	plot(\x, {2*(\x-\hshift)+8+\vshift});
	
	\draw[-,samples=100, very thick, domain=-2+\hshift:2+\hshift]
	plot(\x, {(-2)*(\x-\hshift)+\vshift});
	
	\draw[-, samples=100, very thick, domain=2+\hshift:3+\hshift]
	plot(\x, {(-4)+\vshift});
	
	\draw[-, samples=100, very thick, domain=3+\hshift:4+\hshift]
	plot(\x, {(\x-\hshift)-7+\vshift});
\end{tikzpicture} \begin{enumerate}

\item {\bf (2 points)} $f(\findval)$ \vspace{2cm}

\item {\bf (2 points)} The domain. \vspace{2cm}

\item {\bf (2 points)} The range. \vspace{2cm}

\end{enumerate}


\def \radi{5}\def \circumf{25}

 
\item Suppose the following circle has radius $r= \radi$. What is the circumference of the circle?
\vspace{2mm}

\includegraphics[scale = 0.8]{circle}

\vspace{1cm}
\newpage  $ $   \newpage\end{enumerate}\rhead{10011}\lhead{Sample Course}\chead{Sample Assessment - Fall 2023}\graphicspath{{C:/Users/iainc/anaconda3/Randomizer/Sample Course/Sample Assessment/}}\pagenumbering{arabic}\setcounter{page}{1}


\thispagestyle{fancy}

 
\noindent Name: $\rule{6cm}{0.15mm}$

\vspace{.2cm}

\noindent Student ID: $\rule{6cm}{0.15mm}$

\vspace{.2cm}

\noindent Instructor: $\rule{6cm}{0.15mm}$

\vspace{.2cm}

\noindent Signature: $\rule{6cm}{0.15mm}$
 



\vspace{.4cm}

\noindent {\bf Note that the 'assessmentpreface.tex' file in the exams archive folder is read and placed here. This is also where student information is included, either to be replaced with information from the master.csv file or as blanks.}

\vspace{.4cm}

\hrule

\subsection*{Instructions:} \begin{enumerate}[1.]
\item Any cover page materials, per your departmental standards.
\end{enumerate}


\newpage
\begin{enumerate}
\item {\bf (2 points)} 
 Which of these isn't mentally problematic?

\begin{minipage}[t]{1.0\linewidth}\begin{itemize}\item[(a)]  None of the below.  \item[(b)]  I've built a set that contains itself. \item[(c)]  $a \neq a$ \item[(d)]   All of the above. \end{itemize}\end{minipage} \vfill 
\item {\bf (2 points)} 
 Which of these is correct?

\begin{minipage}[t]{1.0\linewidth}\begin{multicols}{4}\begin{itemize}\item[(a)]  Wrong. \item[(e)]  Wrong. \item[(i)]  Wrong. \item[(b)]  Wrong. \item[(f)]  Wrong. \item[(j)]  Wrong. \item[(c)]  Wrong. \item[(g)]  Wrong. \item[] \item[(d)]  Correct. \item[(h)]  Wrong. \item[] \end{itemize}\end{multicols}\end{minipage} \vfill 
\item {\bf (2 points)} 
 Which of these is correct?

\begin{minipage}[t]{1.0\linewidth}\begin{multicols}{4}\begin{itemize}\item[(a)]  Wrong. \item[(e)]  Correct. \item[(i)]  Wrong. \item[(b)]  Wrong. \item[(f)]  Wrong. \item[(j)]  Wrong. \item[(c)]  Wrong. \item[(g)]  Wrong. \item[] \item[(d)]  Wrong. \item[(h)]  Wrong. \item[] \end{itemize}\end{multicols}\end{minipage} \vfill 
\item {\bf (2 points)} 
 The least common denominator for $\displaystyle \frac{x}{x+1}$ and $\displaystyle \frac{1}{x-1}$ is \vspace{.2cm}

\begin{minipage}[t]{1.0\linewidth}\begin{multicols}{2}\begin{itemize}\item[(a)]  $(x+1)(x-1)$ \item[(c)]  $x(x+1)(x-1)$ \item[(b)]  $x(x+1)$ \item[(d)]  $x+1$ \end{itemize}\end{multicols}\end{minipage} \vfill \newpage\def \a{7}\def \atwoone{2}\def \atwotwo{-2}\def \atwothree{2}\def \btwothree{9}\def \sumtwothree{11}\def \diftwothree{-7}\def \bigtwothree{200}\def \powtwothree{81}\def \logtwothree{0.3154648767857287}\def \factortwothree{187}\def \atwofour{1.47}\def \btwofour{1.342}\def \tooshorttwofour{10.1}\def \moneytwofour{10.10}\def \longertwofour{10.10000}\def \atwofive{0.12}\def \btwofive{0.12346}\def \athreeone{5}\def \bthreeone{4}\def \setthreetwo{[2, 5, 6]}\def \athreetwo{2}\def \bthreetwo{5}\def \cthreetwo{6}\def \controlthreethree{-4}\def \athreethree{3}\def \topthreethree{0}\def \athreefour{3}\def \bthreefour{2}\def \listthreefour{[1, 2, 4, 5]}\def \afourone{16}\def \bfourone{-6}\def \fracfourone{\frac{-8}{3}}\def \rootfourtwo{12}\def \simplifiedfourtwo{2 \sqrt{3}}\def \sqrtlistfourtwo{[2, 3]}\def \outfourtwo{2}\def \infourtwo{3}\def \wowfourtwo{1}\def \afourthree{-5}\def \nicethreefour{3x^{2}-x^{}-5}\def \nastythreefour{xyz^{3}-5}\def \cfourthree{4}\def \dfourthree{9}\def \infourthree{4x^{}}\def \outfourthree{+9y^{}}\def \afourfour{1819277}\def \nicefourfour{1,819,277}\def \goodfourfour{1,000,000.12345}\def \badfourfour{1,000,000.1}
\item {\bf (12 points)} 
 $\athreefour, \bthreefour, \listthreefour$ 
\vfill 
\def \a{7}\def \atwoone{1}\def \atwotwo{-2}\def \atwothree{2}\def \btwothree{9}\def \sumtwothree{11}\def \diftwothree{-7}\def \bigtwothree{200}\def \powtwothree{81}\def \logtwothree{0.3154648767857287}\def \factortwothree{119}\def \atwofour{1.36}\def \btwofour{1.798}\def \tooshorttwofour{10.1}\def \moneytwofour{10.10}\def \longertwofour{10.10000}\def \atwofive{0.12}\def \btwofive{0.12346}\def \athreeone{4}\def \bthreeone{2}\def \setthreetwo{[12, 6, 9]}\def \athreetwo{12}\def \bthreetwo{6}\def \cthreetwo{9}\def \controlthreethree{8}\def \athreethree{4}\def \topthreethree{0}\def \athreefour{5}\def \bthreefour{4}\def \listthreefour{[1, 2, 3, 4]}\def \afourone{16}\def \bfourone{-8}\def \fracfourone{-2}\def \rootfourtwo{8}\def \simplifiedfourtwo{2 \sqrt{2}}\def \sqrtlistfourtwo{[2, 2]}\def \outfourtwo{2}\def \infourtwo{2}\def \wowfourtwo{1}\def \afourthree{0}\def \nicethreefour{3x^{2}-x^{}}\def \nastythreefour{xyz^{3}}\def \cfourthree{4}\def \dfourthree{-10}\def \infourthree{4x^{}}\def \outfourthree{-10y^{}}\def \afourfour{1152056}\def \nicefourfour{1,152,056}\def \goodfourfour{1,000,000.12345}\def \badfourfour{1,000,000.1}
\item {\bf (12 points)} 
 \begin{enumerate}
\item $\atwothree + \btwothree = \sumtwothree$
\item $\atwothree - \btwothree = \diftwothree$
\item $\btwothree^{\atwothree} = \powtwothree$
\item $\atwothree \times 100 = \bigtwothree$
\item $\log_{\btwothree}(\atwothree) = \logtwothree$
\item $\factortwothree$ can be factored into two primes.
\end{enumerate} 
\vfill 
\def \a{7}\def \atwoone{1}\def \atwotwo{3}\def \atwothree{4}\def \btwothree{7}\def \sumtwothree{11}\def \diftwothree{-3}\def \bigtwothree{400}\def \powtwothree{2401}\def \logtwothree{0.7124143742160444}\def \factortwothree{133}\def \atwofour{1.26}\def \btwofour{1.774}\def \tooshorttwofour{10.1}\def \moneytwofour{10.10}\def \longertwofour{10.10000}\def \atwofive{0.12}\def \btwofive{0.12346}\def \athreeone{4}\def \bthreeone{6}\def \setthreetwo{[12, 6, 9]}\def \athreetwo{12}\def \bthreetwo{6}\def \cthreetwo{9}\def \controlthreethree{-4}\def \athreethree{1}\def \topthreethree{0}\def \athreefour{3}\def \bthreefour{2}\def \listthreefour{[1, 2, 4, 5]}\def \afourone{12}\def \bfourone{-8}\def \fracfourone{\frac{-3}{2}}\def \rootfourtwo{20}\def \simplifiedfourtwo{2 \sqrt{5}}\def \sqrtlistfourtwo{[2, 5]}\def \outfourtwo{2}\def \infourtwo{5}\def \wowfourtwo{1}\def \afourthree{5}\def \nicethreefour{3x^{2}-x^{}+5}\def \nastythreefour{xyz^{3}+5}\def \cfourthree{4}\def \dfourthree{9}\def \infourthree{4x^{}}\def \outfourthree{+9y^{}}\def \afourfour{1867725}\def \nicefourfour{1,867,725}\def \goodfourfour{1,000,000.12345}\def \badfourfour{1,000,000.1}
\item {\bf (12 points)} 
 \begin{enumerate}
\item $\atwothree + \btwothree = \sumtwothree$
\item $\atwothree - \btwothree = \diftwothree$
\item $\btwothree^{\atwothree} = \powtwothree$
\item $\atwothree \times 100 = \bigtwothree$
\item $\log_{\btwothree}(\atwothree) = \logtwothree$
\item $\factortwothree$ can be factored into two primes.
\end{enumerate} 
\vfill 
\newpage\def \x{84}\def \y{168}\def \L{336}\def \area{14112}
\item {\bf (5 points)} 
 You have $\L$ feet of fencing to enclose a rectangular plot that borders on a river. If you do not fence along the side of the river, find the \textbf{dimensions} of the plot that will maximize the area. \\

\begin{tikzpicture}[scale=0.2]
    \draw[snake=coil,segment aspect=0,gray] (0,0)   -- (20,0);
    \draw[black] (4,0) -- (4,-4) -- (16,-4) -- (16,0);
    \draw[gray] (10,1.5) node {River};
    \draw[black] (16,-4) node[anchor=west] {Fence};
\end{tikzpicture}
  
\vfill \vfill \vfill
\def \a{5}\def \b{2}\def \c{-7}\def \r{8}\def \monicpol{x^{}+5}\def \longnbad{2x^{2}+3x^{}-27}\def \anspol{2x^{}-7}
\item {\bf (6 points)} 
 Divide the following using {\bf long division}. Your final answer should be in the form $$ \text{Quotient} + \dfrac{\text{Remainder}}{\text{Divisor}}.$$

\vspace{3mm}

$(\longnbad) \div (\monicpol)$

\vfill  \vfill \vfill
\newpage\def \discount{12}\def \paid{1614.90}\def \rainy{10.76}\def \orcost{1835.11}\def \purcost{1441.88}\def \orrainy{12.23}
\item {\bf (5 points)} 
 Elridge Furniture discounts furniture \discount\% to customers paying cash. Jennifer paid \$\paid\ cash for a roll-top desk. What was the original price of the desk? (Round to the nearest cent.) Set up an algebraic equation to represent the situation and solve. Show units.

\vfill 
\def \insvar{24}\def \d{65}\def \zerospeed{39.5}\def \slimit{35}\def \s{53}\def \skidd{117.042}\def \safed{51.042}\def \rsafed{51}

 
\item Solve each and include units in your answer. Use the formula: $s = \sqrt{\insvar \cdot d}$ where $s$ is the speed of the car in miles per hour prior to braking, and $d$ is the stopping distance or length of the skid mark, in feet. 

\vspace{3mm}

Deone is driving down Bumpkin Road going $\slimit$ miles per hour when a fawn suddenly appears $\d$ feet away in the middle of the road. \begin{enumerate}
\item {\bf (5 points)} If she slams on her brakes now, how far will her car skid? \vspace{4cm}
\item {\bf (2 points)} Will she avoid hitting the fawn if it freezes in place? Why or why not? Fully explain your reason. \vspace{3cm}
\end{enumerate}


\newpage\def \xis{5}\def \yis{2}\def \nomatcho{[2,3,5,2]}\def \a{-2}\def \c{3}\def \b{-5}\def \d{-2}\def \polyonesol{-20}\def \polytwosol{11}\def \xgoodone{-2x^{}}\def \ygoodone{-5y^{}}\def \xgoodtwo{3x^{}}\def \ygoodtwo{-2y^{}}\def \unitize{[1,0,0,0,1,0]}\def \mtem{-3}\def \ntem{4}\def \ptem{4}\def \qtem{3}\def \m{1}\def \n{4}\def \p{4}\def \q{3}\def \polytonesol{13}\def \polyttwosol{26}\def \xtgoodone{x^{}}\def \ytgoodone{+4y^{}}\def \xtgoodtwo{4x^{}}\def \ytgoodtwo{+3y^{}}
\item {\bf (6 points)} 
 Solve the system using either substitution or elimination. Write your answer as an ordered pair, if possible. 

$ \begin{cases} \xtgoodone \ytgoodone = \polytonesol \\
\xtgoodtwo \ytgoodtwo = \polyttwosol
\end{cases}$

 \vfill \vfill
\newpage\def \vshift{-1}\def \hshift{2}\def \chang{0}\def \findval{2}\def \yval{-1}

 
\item Given the graph of $f(x)$ below, determine the following. {\bf Assume endpoints are included.}
\vspace{2mm}

\begin{tikzpicture}[scale=0.35]
	\def\startx{-10}
	\def\endx{10}
	\def\starty{-10}
	\def\endy{10}
	
	\draw [very thin,step=1,dotted] (\startx-.4, \starty-.4) grid (\endx+.4, \endy+.4);
	\draw[<->, thick] (\startx-.6,0) -- (\endx+.6, 0);
	\draw[<->, thick] (0,\starty-.6) -- (0,\endy+.6);
	\foreach \x in {\startx,...,\endx}
  	\draw[anchor=north] (\x-0.2, 0) node {\tiny $\x$};
	\foreach \y in {\starty,...,-2,-1,1,2,...,\endy}
  	\draw[anchor=east] (0, \y-.2) node {\tiny $\y$};
  	\draw (0.5, \endy+.3) node {$y$};
  	\draw (\endx+.5, 0.3) node {$x$};
	
	\draw (-4+\hshift,\vshift) node[fill,circle,scale=0.35]{} ;
	\draw (4+\hshift,-3+\vshift) node[fill,circle,scale=0.35]{};
	
	\draw[-, samples=100, very thick, domain=-4+\hshift:-2+\hshift]
	plot(\x, {2*(\x-\hshift)+8+\vshift});
	
	\draw[-,samples=100, very thick, domain=-2+\hshift:2+\hshift]
	plot(\x, {(-2)*(\x-\hshift)+\vshift});
	
	\draw[-, samples=100, very thick, domain=2+\hshift:3+\hshift]
	plot(\x, {(-4)+\vshift});
	
	\draw[-, samples=100, very thick, domain=3+\hshift:4+\hshift]
	plot(\x, {(\x-\hshift)-7+\vshift});
\end{tikzpicture} \begin{enumerate}

\item {\bf (2 points)} $f(\findval)$ \vspace{2cm}

\item {\bf (2 points)} The domain. \vspace{2cm}

\item {\bf (2 points)} The range. \vspace{2cm}

\end{enumerate}


\def \radi{4}\def \circumf{16}

 
\item Suppose the following circle has radius $r= \radi$. What is the circumference of the circle?
\vspace{2mm}

\includegraphics[scale = 0.8]{circle}

\vspace{1cm}
\newpage  $ $   \newpage\end{enumerate} \end{document}